
@unnumbered(Glossary)

@begin(description)
@item Abbrev
An abbrev is a text string which expands into a different text string when
present in the buffer.  For example, you might define a short word as an
abbrev for a long phrase that you want to insert frequently.  @xref[Abbrevs].

@item Aborting
Aborting means getting out of a recursive edit (q.v@.)).  The commands
@kbd{C-]} and @kbd{M-x top-level} are used for this.  @xref[Quitting]. 

@item Auto Fill mode
Auto Fill mode is a minor mode in which text that you insert is
automatically broken into lines of fixed width.  @xref[Filling].

@item Balance Parentheses
Emacs can balance parentheses manually or automatically.
Manual balancing is done by the commands to move over balanced
expressions (@pxref[Lists]).  Automatic balancing is done by
blinking the parenthesis that matches one just inserted
(@pxref[Matching,,Matching Parens]).

@item Bind
To bind a key is to change its binding (q.v@.).

@item Binding
A key gets its meaning in Emacs by having a @dfn{binding} which is a
command (q.v@.), a Lisp function that is run when the key is typed.
Customization often involves rebinding a character to a different
command function.  The bindings of all keys are recorded in the
keymaps (q.v@.).  @xref[Commands,Binding].

@item Blank Lines
Blank lines are lines that contain only whitespace.
Emacs has several commands for operating on the blank lines in
the buffer.

@item Buffer
The buffer is the basic editing unit; one buffer corresponds to one piece
of text being edited.  You can have several buffers, but at any time you
are editing only one, the `selected' buffer, though several can be visible
when you are using multiple windows.  @xref[Buffers].

@item Buffer Selection History
Emacs keeps a buffer selection history which records how recently each
Emacs buffer has been selected.  This is used for choosing a buffer to
select.  @xref[Buffers].

@item C-
@samp{C} in the name of a character is an abbreviation for Control.
@xref[Characters,C-].

@item C-M-
@samp{C-M-} in the name of a character is an abbreviation for
Control-Meta.  @xref[Characters,C-M-].

@item Case Conversion
Case conversion means changing text from upper case to lower case or
vice versa.  @xref[Case], for the commands for case conversion.

@item Characters
Characters form the contents of an Emacs buffer; also, Emacs commands are
invoked by keys (q.v@.), which are sequences of one or more characters.
@xref[Characters].

@item Command
A command is a Lisp function specially defined to be able to serve as an
key binding in Emacs.  When you type a key (q.v@.), its binding (q.v@.) is
looked up in the relevant keymaps (q.v@.) to find the command to run.
@xref[Commands].

@item Command Name
A command name is the name of a Lisp symbol which is a command
(@pxref[Commands]).  You can invoke any command by its name using
@kbd{M-x} (@pxref[M-x]).

@item Comments
A comment is text in a program which is intended only for humans reading
the program, and is marked specially so that it will be ignored when the
program is loaded or compiled.  Emacs offers special commands for creating,
aligning and killing comments.  @xref[Comments].

@item Compilation
Compilation is the process of creating an executable program from source code.
Emacs has commands for compiling files of Emacs Lisp code (@pxref[Lisp
Libraries]) and programs in C and other languages (@pxref[Compilation]).

@item Completion
Completion is what Emacs does when it automatically fills out an
abbreviation for a name into the entire name.  Completion is done
for minibuffer (q.v@.) arguments, when the set of possible valid inputs is
known; for example, on extended command names, buffer names, and file
names.  Completion occurs when @key(TAB), @key(SPC) or @key(RET) is
typed.  @xref[Completion].

@item Continuation Line
When a line of text is longer than the width of the screen, it is takes up
more than one screen line when displayed.  We say that the text line is
continued, and all screen lines used for it after the first are called
continuation lines.  @xref[Basic,Continuation,Basic Editing].

@item Control-Character
ASCII characters with octal codes 0 through 040, and also code 0177, do
not have graphic images assigned to them.  These are the control
characters.  Any control character can be typed by holding down the
@key(CTRL) key and typing some other character; some have special keys on
the keyboard.  @key(RET), @key(TAB), @key(ESC), @key(LFD) and @key(DEL)
are all control characters.  @xref[Characters].

@item Current Buffer
The current buffer in Emacs is the Emacs buffer on which
most editing commands operate.  You can select any Emacs buffer
as the current one.  @xref[Buffers].

@item Current Line
The line point is on (@pxref[Point]).

@item Current Paragraph
The paragraph that point is in.  If point is between paragraphs, the
current paragraph is the one that follows point.  @xref[Paragraphs].

@item Current Defun
The defun (q.v@.) that point is in.  If point is between defuns,
the current defun is the one that follows point.  @xref[Defuns].

@item Cursor
The cursor is the rectangle on the screen which indicates the position
called point (q.v@.) at which insertion and deletion takes place.
Often people speak of `the cursor' when, strictly speaking, they
mean `point'.  @xref[Basic,Cursor,Basic Editing].

@item Customization
Customization is making minor changes in the way Emacs works.  It is often
done by setting variables (@pxref[Variables]) or by rebinding keys
(@pxref[Keymaps]).

@item Default Argument
The default for an argument is the value that will be assumed if you do
not specify one.  When the minibuffer is used to read an argument, the
default argument is used if you just type @key(RET).  @xref[Minibuffer].

@item Default File Name
When a file name arguments is specified, any file name components
that you do not specify are taken from the corresponding components
of the default file name.  In Emacs, the default file name is
normally the name of the file visited in the current buffer.

@item Defun
A defun is a list at the top level of parenthesis or bracket structure in
a program.  It is so named because most such lists in Lisp programs are
calls to the Lisp function @code{defun}.  @xref[Defuns].

@item @key(DEL)
@key(DEL) is a character used as a command to delete one character of
text.  @xref[Basic,Rubout,Basic Editing].

@item Deletion
Deletion means erasing text without saving it.  Emacs deletes text only
when it is expected not to be worth saving (all whitespace, or only one
character).  The alternative is killing (q.v@.).  @xref[Killing,Deletion].

@item Deletion of Files
Deletion of a file means erasing it from the file system.
@xref[Filadv,,Other File Commands].

@item Directory
Files in the Unix file system are grouped into file directories.
@xref[ListDir,,Directories].

@item Dired
Dired is the Emacs facility that displays the contents of a file
directory and allows you to ``edit the directory'', performing
operations on the files in the directory.  @xref[Dired].

@item Disabled Command
A disabled command is one that you may not run without special
confirmation.  The usual reason for disabling a command is that it is
confusing for beginning users.  @xref[Disabling].

@item Dot
An alternate name for `point' (q.v@.), often used in Emacs command names
and on-line documentation.  @xref[Point].

@item Echo Area
The echo area is the bottom line of the screen, used for echoing the
arguments to commands, for asking questions, and printing brief messages
(including error messages).  @xref[Echo Area].

@item Echoing
Echoing is acknowledging the receipt of commands by displaying them
(in the echo area).  Emacs never echoes single-character keys;
longer commands echo only if you pause while typing them.

@item Error Messages
Error messages are single lines of output printed by Emacs when
the user asks for something impossible to do (such as, killing text
forward when point is at the end of the buffer).  They appear
in the echo area, accompanied by a beep.

@item @key(ESC)
@key(ESC) is a character, used to end incremental searches and as a prefix
for typing Meta characters on keyboards lacking a @key(META) key.

@item Fill Prefix
The fill prefix is a string that should be expected at the beginning
of each line when filling is done.  It is not regarded as part of the
text to be filled.  @xref[Filling].

@item Filling
Filling text means moving text from line to line so that all the lines
are approximately the same length.  @xref[Filling].

@item Global
Global means `independent of the current environment; in effect
throughout Emacs'.  It is the opposite of local (q.v@.).  Particular
examples of the use of `global' appear below.

@item Global Abbrev
A global definition of an abbrev (q.v@.) is effective in all major modes
that do not have local (q.v@.) definitions for the same abbrev.
@xref[Abbrevs].

@item Global Keymap
The global keymap (q.v@.) contains key bindings that are in effect except
when overriden by local key bindings in a major mode's local keymap (q.v@.).
@xref[Keymaps].

@item Global Substitution
Global substitution means replacing one string by another string through a
large amount of text.  @xref[Replace].

@item Global Variable
The global value of a variable (q.v@.) takes effect in all buffers that do
not have their own local (q.v@.) values for the variable.  @xref[Variables].

@item Graphic Character
Graphic characters are those assigned pictorial images rather than just
names.  All the non-Meta (q.v@.) characters except for the Control (q.v@.)
characters are graphic characters.  These include letters, digits,
punctuation, and spaces; they do not include @key(RET) or @key(ESC).  In
Emacs, typing a graphic character inserts that character.
@xref[Basic,,Basic Editing].

@item Grinding
Grinding means reformatting a program so that it is indented
according to its structure.  @xref[Indentation,Grinding].

@item Hardcopy
Hardcopy means printed output.  Emacs has commands for making
printed listings of text in Emacs buffers.  @xref[Hardcopy].

@item @key(HELP)
You can type @key(HELP) at any time to ask what options you have, or to
ask what any command does.  @key(HELP) is really @kbd{Control-h}.
@xref[Help].

@item Indentation
Indentation means blank space at the beginning of a line.  Most
programming languages have conventions for using indentation to
illuminate the structure of the program, and Emacs has special
features to help you set up the correct indentation.
@xref[Indentation].

@item Insertion
Insertion means copying text into the buffer, either from the
keyboard or from some other place in Emacs.

@item Justification
Justification means adding extra spaces to lines of text to make
them come exactly to a specified width.
@xref[Filling,Justification].

@item Keyboard Macros
Keyboard macros are a way of defining new Emacs commands
from sequences of existing ones, with no need to write a Lisp
program.  @xref[Keyboard Macros].

@item Key
A key is a character or sequence of characters which, when typed by
the user, fully specifies one action to be performed by Emacs.  For
example, @kbd{X} and @kbd{Control-f} and @kbd{Control-x m} are keys.
Keys derive their meanings from being bound (q.v@.) to commands (q.v@.).
@xref[Keys].

@item Keymap
The keymap is the data structure that records the bindings (q.v@.)  of
keys to the commands that they run.  For example, the keymap binds
the character @kbd{C-n} to the command function @code{next-line}.
@xref[Keymaps].

@item Kill Ring
The kill ring is where all text you have killed recently is saved.
You can reinsert any of the killed text still in the ring; this is called
yanking (q.v@.).  @xref[Yanking].

@item Killing
Killing means erasing text and saving it on the kill ring so it can be
yanked (q.v@.) later.  Most Emacs commands to erase text do killing, as
opposed to deletion (q.v@.).  @xref[Killing].

@item List
A list is, approximately, a text string beginning with an open
parenthesis and ending with the matching close parenthesis.  In C mode
and other non-Lisp mode groupings surrounded by other kinds of matched
delimiters appropriate to the language, such as braces, are also
considered lists.  Emacs has special commands for many operations on
lists.  @xref[Lists].

@item Local
Local means `in effect only in a particular context'; the relevant kind of
context is a particular function execution, a particular buffer, or a
particular major mode.  It is the opposite of `global' (q.v@.).  Specific
uses of `local' in Emacs terminology appear below.

@item Local Abbrev
A local abbrev definition is effective only if a particular major mode is
selected.  In that major mode, it overrides any global definition for the
same abbrev.  @xref[Abbrevs].

@item Local Keymap
A local keymap is used in a particular major mode; the key bindings (q.v@.)
in the current local keymap override global bindings of the same keys.
@xref[Keymaps].

@item Local Variable
A local value of a variable (q.v@.) applies to only one buffer.
@xref[Variables].

@item M-
@kbd{M-} in the name of a character is an abbreviation for @key(META),
one of the modifier keys that can accompany any character.
@xref[Characters].

@item M-x
@kbd{M-x} is the key which is used to call an Emacs command by name.
This is how commands that are not bound to keys are called.  @xref[M-x].

@item Mail
Mail means messages sent from one user to another through the computer
system.  Emacs has commands for composing and sending mail, and for reading
and editing the mail you have received.  @xref[Mail].

@item Major Mode
The major modes are a mutually exclusive set of options each of which
configures Emacs for editing a certain sort of text.  Ideally, each
programming language has its own major mode.  @xref[Major Modes].

@item Mark
The mark points to a position in the text.  It specifies one end of the
region (q.v@.), point being the other end.  Many commands operate on
all the text from point to the mark.  @xref[Mark].

@item Mark Ring
The mark ring is used to hold several recent previous locations of
the mark, just in case you want to move back to them.  @xref[Mark Ring].

@item Message
See `mail'.

@item Meta
Meta is the name of a modifier bit which a command character may have.  It
is present in a character if the character is typed with the @key(META) key
held down.  Such characters are given names that start with @kbd{Meta-}.
For example, @kbd{Meta-<} is typed by holding down @key(META) and typing
@kbd{<} (which itself is done by holding down @key(SHIFT) and typing
@kbd{,}).  @xref[Characters,Meta].

@item Meta Character
A Meta character is one whose character code includes the Meta bit.

@item Minibuffer
The minibuffer is the window that appears when necessary inside the echo
area (q.v@.), used for reading arguments to commands.  @xref[Minibuffer].

@item Minor mode
A minor mode is an optional feature of Emacs which can be switched on or
off independently of all other features.  Each minor mode has a command to
turn it on or off.  @xref[Minor Modes].

@item Mode line
The mode line is the line at the bottom of each text window (q.v@.), which
gives status information on the buffer displayed in that window.
@xref[Mode Line].

@item Modified Buffer
A buffer (q.v@.) is modified if its text has been changed since the last
time the buffer was saved (or when it was created, if it has never been
saved).  @xref[Saving].

@item Moving Text
Moving text means erasing it from one place and inserting it in another.
This is done by killing (q.v@.) and then yanking (q.v@.).  @xref[Killing].

@item Named Mark
A named mark is a register (q.v@.) in its role of recording a location in
text so that you can move point to that location.  @xref[Registers].

@item Narrowing
Narrowing means creating a restriction (q.v@.) that limits editing
in the current buffer to only a part of the text in the buffer.  Text
outside that part is inaccessible to the user until the boundaries are
widened again, but it is still there, and saving the file saves it all.
@xref[Narrowing].

@item Newline
@key(LFD) characters in the buffer terminate lines of text and are
called newlines.  @xref[Characters,Newline].

@item Numeric Argument
A numeric argument is a number, specified before a command, to change the
effect of the command.  Often the numeric argument serves as a repeat
count.  @xref[Arguments].

@item Option
An option is a variable (q.v@.) that exists so that you can customize
Emacs by giving it a new value.  @xref[Variables].

@item Overwrite Mode
Overwrite mode is a minor mode.  When it is enabled, ordinary text
characters replace the existing text after point rather than pushing it to
the right.  @xref[Minor Modes].

@item Page
A page is a unit of text, delimited by formfeed characters (ASCII @ctl[L],
code 014) coming at the beginning of a line.  Some Emacs commands are
provided for moving over and operating on pages.  @xref[Pages].

@item Paragraphs
Paragraphs are the medium-size unit of English text.  There are special
Emacs commands for moving over and operating on paragraphs.
@xref[Paragraphs].

@item Parsing
We say that Emacs parses words or expressions in the text being edited.
Really, all it knows how to do is find the other end of a word or
expression.  @xref[Syntax].

@item Point
Point is the place in the buffer at which insertion and deletion occur.
Point is considered to be between two characters, not at one character.
The terminal's cursor (q.v@.) indicates the location of point.
@xref[Basic,Point].

@item Prefix Key
A prefix key is a key (q.v@.) whose sole function is to introduce a set of
multi-character keys.  @kbd{Control-x} is an example of prefix key; thus,
any sequence of @kbd{C-x} followed by one other character is also a
legitimate key.  @xref[Keys].

@item Prompt
A prompt is text printed to ask the user for input.  Printing a prompt is
called @dfn{prompting}.  Emacs prompts always appear in the echo area
(q.v@.).  One kind of prompting happens when the minibuffer is used to read
an argument (@pxref[Minibuffer]); the echoing which happens when you pause
in the middle of typing a multicharacter key is also a kind of
prompting (@pxref[Echo Area]).

@item Quitting
Quitting means cancelling a partially typed command or a running command,
using @kbd{C-g}.  @xref[Quitting].

@item Quoting
Quoting means depriving a character of its usual special significance.  It
is usually done with @kbd{Control-q}.  What constitutes special
significance depends on the context and on convention.  For example, an
``ordinary'' character as an Emacs command inserts itself; so in this
context, a special character is any character that does not normally insert
itself (such as @key(DEL), for example), and quoting it makes it insert
itself as if it were not special.  Not all contexts allow quoting.
@xref[Basic,Quoting,Basic Editing].

@item Read-only Buffer
A read-only buffer is one whose text you are not allowed to change.
Normally Emacs makes buffers read-only when they contain text which has a
special significance to Emacs; for example, Dired buffers.  Visiting a file
that is write protected also makes a read-only buffer.  @xref[Buffers].

@item Recursive Editing Level
A recursive editing level is a state in which part of the execution of a
command involves asking the user to edit some text.  This text may or may
not be the same as the text to which the command was applied.  The mode
line indicates recursive editing levels with square brackets (@samp{[} and
@samp{]}).  @xref[Recursive Edit].

@item Redisplay
Redisplay is the process of correcting the image on the screen to
correspond to changes that have been made in the text being edited.
@xref[Screen,Redisplay].

@item Region
The region is the text between point (q.v@.) and the mark (q.v@.).  Many
commands operate on the text of the region.  @xref[Mark,Region].

@item Registers
Registers are named slots in which text or buffer positions or rectangles
can be saved for later use.  @xref[Registers].

@item Replacement
See `global substitution'.

@item Restriction
A restriction in a buffer makes some of the text at the beginning
or end of the buffer, or both, temporarily invisible and inaccessible.
Creating a restriction on a buffer is called narrowing, and
removing one is called widening.  @xref[Narrowing].

@item @key(RET)
@key(RET) is an Emacs command to insert a newline into the text.
It is also used to terminate most arguments read in the minibuffer (q.v@.).
@xref[Characters,Return].

@item Saving
Saving a buffer means copying its text into the file that was visited
(q.v@.) in that buffer.  This is the way text in files actually gets
changed by your Emacs editing.  @xref[Saving].

@item Scrolling
Scrolling means shifting the text in the Emacs window so as to see a
different part of the buffer.  @xref[Display,Scrolling].

@item Searching
Searching means moving point to the next occurrence of a specified string.
@xref[Search].

@item Selecting
Selecting a buffer means making it the current (q.v@.) buffer.
@xref[Buffers,Selecting].

@item Self-documentation
Self-documentation is the feature of Emacs which can tell you what any
command does, or give you a list of all commands related to a topic you
specify.  You ask for self-documentation with the @key(HELP) character.
@xref[Help].

@item Sentences
Emacs has commands for moving by or killing by sentences.
@xref[Sentences].

@item Sexp
A sexp (short for `s-expression') is the basic syntactic unit of Lisp in
its textual form: either a list, or Lisp atom.  Many Emacs commands operate
on sexps.  The term `sexp' is generalized to languages other than Lisp,
to mean a syntactically recognizable expression.
@xref[Lists,Sexps].

@item Simultaneous Editing
Simultaneous editing means two users modifying the same file at once.
Simultaneous editing if not detected can cause one user to lose his work.
Emacs detects all cases of simultaneous editing and warns the user to
investigate them.  @xref[Interlocking,,Simultaneous Editing].

@item String Substitution
See `global substitution'.

@item Syntax Table
The syntax table tells Emacs which characters are part of a word, which
characters balance each other like parentheses, etc.  @xref[Syntax].

@item Tag Table
A tag table is a file that serves as an index to the function
definitions in one or more other files.  @xref[Tags].

@item Text
Two meanings (@pxref[Text]):
@itemize @bullet
@item
Data consisting of a sequence of characters.  The contents of an
Emacs buffer are always text in this sense.
@item
Data consisting of written human language, as opposed to programs, or
following the stylistic conventions of human language.
@end itemize

@item Top Level
Top level is the normal state of Emacs, in which you are editing the text
of the file you have visited.  You are at top level whenever you are not in
a recursive editing level (q.v@.) or the minibuffer (q.v@.), and not in the
middle of a command.  You can get back to top level by aborting (q.v@.) and
quitting (q.v@.).  @xref[Quitting].

@item Transposition
Transposing two units of text means putting each one into the place
formerly occupied by the other.  There are Emacs commands to transpose two
adjacent characters, words, sexps (q.v@.) or lines (@pxref[Transposition]).

@item Truncation
Truncating text lines in the display means leaving out any text on a line
that does not fit within the right margin of the window displaying it.
See also `continuation line'.  @xref[Basic,Truncation,Basic Editing].

@item Undoing
Undoing means making your previous editing go in reverse, bringing back the
text that existed earlier in the editing session.  @xref[Undo].

@item Variable
A variable is an object in Lisp that can store an arbitrary value.
Emacs uses some variables for internal purposes, and has others (known as
`options' (q.v@.)) just so that you can set their values to control the
behavior of Emacs.  The variables used in Emacs that you are likely to be
interested in are listed in the Variables Index in this manual.
@xref[Variables], for information on variables.

@item Visiting
Visiting a file means loading its contents into a buffer (q.v@.) where they
can be edited.  @xref[Visiting].

@item Wall Chart
The wall chart is a very brief Emacs reference sheet giving one line of
information about each short command.
@iftex
A copy of the wall chart appears in this manual.
@end iftex

@item Whitespace
Whitespace is any run of consecutive formatting characters (space, tab,
newline, and backspace).

@item Widening
Widening is removing any restriction (q.v@.) on the current buffer; it is
the opposite of narrowing (q.v@.).  @xref[Narrowing].

@item Window
Emacs divides the screen into one or more windows, each of which can
display the contents of one buffer (q.v@.) at any time.  @xref[Screen], for
basic information on how Emacs uses the screen.  @xref[Windows], for
commands to control the use of windows.

@item Word Abbrev
Synonymous with `abbrev'.

@item Word Search
Word search is searching for a sequence of words, considering the
punctuation between them as insignificant.  @xref[Word Search].

@item Yanking
Yanking means reinserting text previously killed.  It can be used to undo a
mistaken kill, or for copying or moving text.  @xref[Yanking].
@end description
