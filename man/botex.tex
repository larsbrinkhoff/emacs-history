%% TeX macros to simulate something like Bolio
\def\BoTeXrevdate{15 July 1985} \def\BoTeXversion{0.26}
\message{Loading BoTeX package [Version \BoTeXversion\space of \BoTeXrevdate]:}
\message{}
%% Revisions:
%%
%% Version    Date    Changes
%%   0.02  31 Oct 84  (RMS) \Bfoo -> \foo, argument parsing, Scribble cruft,
%%			gobs of features.
%%   0.03   2 Nov 84  (RpK) Try to get headings to work; use CM fonts,
%%			centralize outside font definitions.
%%   0.04   6 Nov 84  (RpK) Tune the fonts, add @b,i,r,t from Scribe.
%%			Plus:	@l  Lisp font, usually sans-serif)
%%				@s  Small caps (can't use @c !)
%%				@ii TeX italics.  (@i is really slanted.)
%%   0.05  13 Nov 84  (RpK) @titlepage command, @copyrightpage environment
%%			@setq x section-name, chapter-name
%%   0.06  14 Nov 84  (RpK) Headings are off by default; turn them on after
%% 			the front pages.  @end[foo] is smart about error
%%			message -- top level \Efoo's are no longer needed.
%%   0.07  20 Nov 84  (RpK) Kill @section page break.
%%   0.08  21 Nov 84  (RpK) Fix headings (again, sigh)
%%   0.09  27 Nov 84  (RpK) Make headings a little smaller.  Avoid 
%%			@thischapter lookahead lossage.
%%   0.10  29 Nov 84  (RpK) Fix @setx, increment section #'s globally
%%   0.11  30 Nov 84  (RpK) Kludge for @@ until the fonts are fixed.
%%   0.12   6 Dec 84  (RpK) @fontvector, @settabs support, @bye shows stats,
%%			() as delims, itemindent is 8.in, @end description
%%			@lisp is less fussy (more \hfuzz), @label and
%%			@pageref, @title, and @nameref for it.
%%   0.13  10 Dec 84  (RpK) @figure, @fullpage-, @top-.  No toc for them yet.
%%   0.13a 17 Dec 84  (RpK) Typo in odd-page chapter macro
%%   0.14  15 Jan 85  (RpK) @synindex bug.  @cindex &c. uses \parsearg
%%   0.15  23 Jan 85  (RpK) Fix output, adding \onepageout.
%%   0.16   7 Mar 85  (RpK) Put lmipart on front, add copyrightdate variable
%%   0.18   4 Apr 85  (RpK) ``open'' chapter style started.  More
%%                      space before a section heading; \subsecheadingskip
%%			Fix output (again !); appendix letters in TOC.
%%   0.19  23 Apr 85  (RpK) Fix headers/footers in indices.  Fix typos.
%%   0.20   5 Jun 85  (RMS) {} delimited arguments.  ``No Title''
%%			is the initial title.
%%   0.21  15 Jun 85  (RMS) Use \aboveenvbreak and \afterenvbreak
%%			rather than \smallbreak, around environments.
%%			\aboveenvbreak adds in \parskip to make space
%%			after and before the same.
%%			\aboveenvskipamount is amount of additional skip
%%			beyond \parskip.
%%			Add in \parskip to break before section heading.
%%			Make penalty after section heading really work.
%%			\parskip inside itemize, etc, is same as
%%			item spacing. 	Define @example, same as @lisp.
%%			Make xrefs get section # when read, not
%%			when shippedout.
%%   0.22 17 Jun 85  (RpK) Snag runaway intersection whitespace
%%   0.23 19 Jun 85  (RMS) Fix unaligned dots in indexes.
%%			Fix special characters in indexes.
%%   0.24 13 Jul 85  (RMS) Define \tableindent, indent of table
%%			bodies.  \itemindent still used for
%%			@itemize and @enumerate.
%%			Use \aboveenvbreak and \afterenvbreak
%%			in @table, @itemize, @enumerate.
%%   0.25 14 Jul 85  (RMS) Define @subsubheading.
%%			Fix bug where aboveenvbreak clobbered
%%			aboveenvskipamount.
%%   0.26 15 Jul 85  (RMS) Use \medbreak between chapters in
%%			@summarycontents.  Double indentation
%%			for sections, subsecs, subsubsecs
%%			in full contents.
%% Todo:
%% Tabular/Matrix environments, figures
%% Fix @, \ characters (fonts are generally 'orrible)

\message{Basics,}
\chardef\other=12
\parskip=1.5pt
\advance\topskip by 1.2cm
% Compensate for losing imagen.

\advance \hoffset 1in
\advance \voffset 1in

\hyphenation{ap-pen-dix zeta-lisp}

% Copyright notice
\def\Xcopyrightdate{\number\year}
\def\LMIcopyright{Copyright \copyright\ \Xcopyrightdate\ Lisp Machine Incorporated.}
\def\LMIpub{\vbox{%
\line{\hfil Published by LMI 1000 Massachusetts Avenue.  Cambridge MA 02138 USA}}}

% Margin to add to right of even pages, to left of odd pages.
\newdimen \bindingoffset  \bindingoffset=0pt
\newdimen \normaloffset   \normaloffset=\hoffset
\newdimen\pagewidth \newdimen\pageheight
\pagewidth=\hsize \pageheight=\vsize

% \onepageout takes a vbox as an argument.  Note that \pagecontents
% does insertions itself, but you have to call it yourself.
\chardef\PAGE=255  \output={\onepageout{\pagecontents\PAGE}}
\def\onepageout#1{\hoffset=\normaloffset
\ifodd\pageno  \advance\hoffset by \bindingoffset
\else \advance\hoffset by -\bindingoffset\fi
\shipout\vbox{{\let\hsize=\pagewidth \makeheadline} \pagebody{#1}%
 {\let\hsize=\pagewidth \makefootline}}
\advancepageno \ifnum\outputpenalty>-20000 \else\dosupereject\fi}
\def\pagebody#1{\vbox to\pageheight{\boxmaxdepth=\maxdepth #1}}
{\catcode`\@ =11
\gdef\pagecontents#1{\ifvoid\topins\else\unvbox\topins\fi
\dimen@=\dp#1 \unvbox#1
\ifvoid\footins\else\vskip\skip\footins\footnoterule \unvbox\footins\fi
\ifr@ggedbottom \kern-\dimen@ \vfil \fi}
}

% Parse an argument, then pass it to #1.
% The argument can be delimited with [...] or with "..."
% or it can be a whole line.
% #1 should be a macro which expects
% an ordinary undelimited TeX argument.

\def\parsearg #1{\let\next=#1\begingroup\obeylines\futurelet\temp\parseargx}

\def\parseargx{%
\ifx \obeyedspace\temp \aftergroup\parseargdiscardspace \else%
\ifx [\temp \aftergroup\parseargbracket\else%
\ifx (\temp \aftergroup\parseargparen\else%
\ifx \activedoublequote\temp \aftergroup\parseargdoublequote\else%
\ifx \mylbrace\temp \aftergroup\parseargbrace\else%
\aftergroup \parseargline %
\fi\fi\fi\fi\fi \endgroup}

{\obeyspaces %
\gdef\parseargdiscardspace {\begingroup\obeylines\futurelet\temp\parseargx}}

\gdef\obeyedspace{\ }

\def\parseargbracket [#1]{\next {#1}} \def\parseargparen (#1){\next {#1}}
{\catcode `\"=\active
\gdef\parseargdoublequote "#1"{\next {#1}}}

{\catcode `\[=1 \catcode `\]=2 \catcode `\}=\active
\catcode `\{=\active
\gdef\parseargbrace {#1}[\next [#1]]]

\def\parseargline{\begingroup \obeylines \parsearglinex}
{\obeylines %
\gdef\parsearglinex #1^^M{\endgroup \next {#1}}}

%% These are used to keep @begin/@end levels from running away
%% Call \inENV within environments (after a \begingroup)
\newif\ifENV \ENVfalse \def\inENV{\ifENV\relax\else\ENVtrue\fi}
\def\ENVcheck{%
\ifENV\errmessage{Still within an environment.  Type Return to continue.}
\endgroup\fi} % This is not perfect, but it should reduce lossage

% @begin foo  is the same as @foo, for now.
\newhelp\EMsimple{Type <Return> to continue}

\outer\def\begin{\parsearg\beginxxx}

\def\beginxxx #1{%
\expandafter\ifx\csname #1\endcsname\relax
{\errhelp=\EMsimple \errmessage{Undefined command @begin #1}}\else
\csname #1\endcsname\fi}

%% @end foo executes the definition of \Efoo.
%% foo can be delimited by doublequotes or brackets.
\let\ptexend = \end

\def\end{\parsearg\endxxx}

\def\endxxx #1{%
\expandafter\ifx\csname E#1\endcsname\relax
\expandafter\ifx\csname #1\endcsname\relax
\errmessage{Undefined command @end #1}\else
\errorE{#1}\fi\fi
\csname E#1\endcsname}
\def\errorE#1{
{\errhelp=\EMsimple \errmessage{@end #1 not within #1 environment}}}

% Simple aliases that make some plain tex constructs available.

\let\nopara=\noindent

\let\ptexnobreak=\nobreak
\def\nobreak{\par\penalty 10000}

% Single-spacing is done by various environments.

\newskip\singlespaceskip \singlespaceskip = \baselineskip
\def\singlespace{\baselineskip=\singlespaceskip}

% @@ prints an @ -- for Scribe compatibility.
% Kludge this until the fonts are right (grr).
\def\@{{\sf @}}

% @: is a Scribe command to cause end-of-sentence whitespace.
\def\:{\spacefactor=3000 }

% @* forces a line break.
\let\ptexstar=\* \def\*{\hfil\break}

% @. is an abbreviation period.
\let\ptexdot=\. \def\.{.\spacefactor=1000 }

% @# leaves space for a special character.
\let\ptexnumsign=\# \def\#{\write19{(@##)}\hbox to 0.7em{\hfil}}

% @w prevents a word break
\def\w{\parsearg\atomx} \def\atomx #1{\hbox{#1}}

% Save the essence of & for tabular environments as @\ for BoTeX
% Say @settabs 4 @columns, then @<col0@\col1@\col2@\col3@cr
\let\\=& \let\<=\+
% @+ raises its argument, @- lowers.
\let\ptexplus=\+
\def\+{\parsearg\plusxxx} \def\plusxxx #1{\raise 1ex\hbox{#1}}

\let\ptexminus=\-
\def\-{\parsearg\minusxxx} \def\minusxxx #1{\lower 1ex\hbox{#1}}

% @titlepage outputs a standard LMI window title page
% titlecomment, lmipart can be @setx'ed
\def\Xlmipart{} \def\Xtitlecomment{}
\def\titlepage{
{\advance\hsize by 0.064truein % Just enough to hit the window
\line{\hfil \hbox{
 \vbox to 3.02truein{
  \vfil
  \hrule\kern 2truept\hrule\kern 3.5truept
  \vbox to 1.4truein{
   \vfil
   \hbox to 4truein{\hfil{\chapbf\thistitle}\hfil}
   \hbox to 4truein{\hfil{\btwelve\Xtitlecomment}\hfil}
   \vfil
   \hbox to 4truein{\hfil{\tenrm\Xlmipart}}}
  \kern 3.5truept\hrule\kern 2truept\hrule}}}
\par\vfill \LMIpub \supereject
}} % end \def

% @copyrightpage smushes things toward the bottom
\def\copyrightpage{\begingroup\inENV
\def\Ecopyrightpage{\par\kern14pt\hrule%
\kern14pt\line{\LMIcopyright\hfil}\egroup\endgroup\supereject\HEADINGSon}%
\vbox to\vsize \bgroup \vfill}

% @group ... @end group  forces ... to be all on one page.

\def\group{\begingroup% \inENV ???
\def \Egroup{\egroup\endgroup}
\vbox\bgroup}

% @br   forces paragraph break

\let\br = \par

% @page    forces the start of a new page

\def\page{\par\vfill\supereject}

% @exdent n text....
% outputs text on separate line in roman font, starting at standard page margin
% The argument n is ignored.  This is most likely to be right
% for the ways @exdent actually appears in Bolio files.

\def\exdent{\errmessage{@exdent in filled text}}
  % @lisp, etc, define \exdent locally from \internalexdent

{\obeyspaces
\gdef\internalexdent #1 {\parsearg\exdentzzz}}

\def\exdentzzz #1{{\advance \leftskip by -\lispnarrowing
\advance \hsize by -\leftskip
\advance \hsize by -\rightskip
\leftline{{\rm#1}}}}

% @include file    insert text of that file as input.

\def\include{\parsearg\includexxx}

\def\includexxx #1{{\def\thisfile{#1}\input #1
}}

\def\thisfile{}

% @need space-in-mils
% forces a page break if there is not space-in-mils remaining.

\newdimen\mil  \mil=0.001in

\def\need{\parsearg\needx}

\def\needx #1{\par %
\begingroup %
\dimen0=\pagetotal %
\advance \dimen0 by #1\mil %
\ifdim \dimen0>\pagegoal \vfill\eject \fi %
\endgroup}

% @setpageheight and @setpagewidth
% These are no longer really necessary
% since you can now do @hsize = or @vsize =

\def\setpagewidth #1 {\global\hsize #1}
\def\setpageheight #1 {\global\vsize #1}

% @center line   outputs that line, centered

\def\center{\parsearg\centerxxx}

\def\centerxxx #1{{\advance\hsize by -\leftskip
\advance\hsize by -\rightskip
\centerline{#1}}}

% @sp n   outputs n lines of vertical space

\def\sp{\parsearg\spacexxx}

\def\spacexxx #1{\par \vskip #1\baselineskip}

% @comment ...line which is ignored...
% @c is the same as @comment
% @ignore ... @end ignore  is another way to write a comment

\def\comment{\parsearg \commentxxx}

\def\commentxxx #1{}

\let\c=\comment

\long\def\ignore #1\end ignore{}

% Document-version conditionals
\message{conditionals,}

% @setflag foo   sets flag "foo"
% @clearflag foo   clears flag "foo"
% @defaultsetflag foo    sets flag if never been explicitly set or cleared
% @ifset [foo]...@end if  includes body if foo is set
% @ifclear [foo]...@end if   includes body if foo is not set.
% A flag is initially clear by default.

% A flag is represented by the value of a control sequence
% whose name is F followed by the flag name.
% If its definition is \relax, it has never been explicitly set or cleared.
% This counts as "clear" for everything except @defaultsetflag.
% It is the initial state.
% Flags explicitly set or cleared have these definitions:
\def\flagtrue{true}
\def\flagfalse{false}

\def\setflag{\parsearg\setflagxxx}
\def\setflagxxx #1{\expandafter\let\csname F#1\endcsname=\flagtrue}

\def\clearflag{\parsearg\clearflagxxx}
\def\clearflagxxx #1{\expandafter\let\csname F#1\endcsname=\flagfalse}

\def\defaultsetflag{\parsearg\defaultsetflagxxx}
\def\defaultsetflagxxx #1{%
\expandafter\ifx\csname F#1\endcsname\relax
\expandafter\let\csname F#1\endcsname=\flagtrue
\fi}

\def\ifset{\parsearg\ifsetxxx}

\def\ifsetxxx #1{%
\expandafter\ifx\csname F#1\endcsname\flagtrue \let\next=\relax \else
\let\next=\iffails \fi \next}

\def\ifnotset{\parsearg\ifnotsetxxx}
\def\ifclear{\parsearg\ifnotsetxxx}

\def\ifnotsetxxx #1{%
\expandafter\ifx\csname F#1\endcsname\flagtrue \let\next=\iffails \else
\let\next=\relax \fi \next}

\def\iffails #1\end if{}
\def\Eif{}

\message{fonts,}
% Font-change commands.
\font\btwelve=ambx10 at 12pt
%% Try out Computer Modern fonts at a little bigger than \magstephalf
\font\tenrm=cmr10 scaled 1100
\font\tentt=amtt10 scaled \magstephalf
\font\tenbf=cmb10 scaled 1100
\font\tenit=cmti10 scaled 1100
\font\tensl=cms10 scaled 1100
% \font\tensf=cmss10 scaled 1100 
\font\tensf=amss10 scaled 1100
\def\li{\sf}
\font\tensc=amcsc10 scaled \magstephalf

\font\defbf=ambx7 scaled \magstep2
\let\deftt=\tentt
\font\twelvesf=cmss12

% Font for title
\font\titlerm = cmb10 scaled 2800

% Fonts for indices
\font\indit=cmi9 \font\indrm=cmr9
\def\indbf{\indrm} \def\indsl{\indit}
\def\indexfonts{\let\it=\indit \let\sl=\indsl \let\bf=\indbf \let\rm=\indrm}

\font\secrm=cmb10 scaled 1400
\font\secit=cmti10 scaled 1400
\font\secsl=cms10 scaled 1400
\let\secbf=\secrm

\font\chaprm=cmb10 scaled 1800
\font\chapit=cmti10 scaled 1800
\let\chapbf=\chaprm

\font\ssecrm=cmb10 scaled 1200
\font\ssecit=cmti10 scaled 1200
\font\ssecsl=cms10 scaled 1200
\let\ssecbf=\ssecrm

\def\textfonts{\let\rm=\tenrm\let\it=\tenit\let\sl=\tensl\let\bf=\tenbf%
\let\sc=\tensc\let\sf=\tensf}
\def\chapfonts{\let\rm=\chaprm\let\it=\chapit\let\sl=\chapsl\let\bf=\chapbf}
\def\secfonts{\let\rm=\secrm\let\it=\secit\let\sl=\secsl\let\bf=\secbf}
\def\subsecfonts{\let\rm=\ssecrm\let\it=\ssecit\let\sl=\ssecsl\let\bf=\ssecbf}
% Count depth in font-changes, for error checks
\newcount\fontdepth \fontdepth=0

\def\fontspec #1{\endgroup
\if #1$\epsilon$\else \if #1*\fontpop \global\advance\fontdepth by -1 \else
\global\advance\fontdepth by 1 %
\begingroup\let\fontpop=\endgroup\fontselect#1\fi\fi}

%% This is zero-based; things actually start out in 0 (=rm by default) now.
%% Font 9 is always \titlerm for hysterical reasons
\def\nofont{\errmessage{No such font.  Type Return to continue.}}
\def\fontvector#1#2#3#4#5#6#7#8#9{
\def\FMzero{#1} \def\FMone{#2} \def\FMtwo{#3} \def\FMthree{#4} \def\FMfour{#5}
\def\FMfive{#6} \def\FMsix{#7} \def\FMseven{#8} \def\FMeight{#9}}
\fontvector \rm \bf \sl \li \chaprm \chaprm \nofont \tt \nofont
%% Ever hear of AREF ? Actually, \ifcase seems to lose here
\def\fontselect#1{\if #10\FMzero\else\if #11\FMone\else\if#12\FMtwo\else%
\if #13\FMthree\else\if #14\FMfour\else\if #15\FMfive\else\if #16\FMsix\else%
\if #17\FMseven\else\if #18\FMeight\else\if #19\titlerm
\else\errmessage{Illegal font selector (#1)}\fi\fi\fi\fi\fi\fi\fi\fi\fi\fi}

\def\fontpop{\errmessage{extra ^F*}}

\def\levelcheck{\ifnum \fontdepth > 0 %
\errmessage{Unmatched font-changes before this point}\fi\ENVcheck}

\catcode`\^^F=\active
\def\ctlf{\begingroup\catcode`\^^F=\other\fontspec}
\let^^F=\ctlf

%% Add scribe-like font environments, plus @l for inline lisp (usually sans
%% serif) and @ii for TeX italic
\let\ptexb=\b \let\ptexc=\c \let\ptexi=\i \let\ptext=\t
\let\ptexl=\l \let\ptexL=\L
\def\I{\parsearg\iF} \let\i=\I \def\iF#1{{\sl #1}}
\def\R{\parsearg\rF} \let\r=\R \def\rF#1{{\rm #1}}
\def\B{\parsearg\bF} \let\b=\B \def\bF#1{{\bf #1}}
\def\L{\parsearg\lF} \let\l=\L \def\lF#1{{\li #1}}
\def\s{\parsearg\sF}	       \def\sF#1{{\sc #1}}
\def\T{\parsearg\tF} \let\t=\T \def\tF#1{{\tt #1}}
\def\II{\parsearg\iiF} \let\ii=\II \def\iiF#1{{\it #1}}

% ^Q quote next character
\catcode`\^^Q=\active

\def\ctlq{\begingroup
\catcode `\@=\other 
\afterassignment\quotechar\let\nextchar= }

\let^^Q=\ctlq

\def\quotechar{%
\ifx \nextchar\ref $\otimes$\else
\ifx \nextchar\ctlq $\supset$\else
\ifx \nextchar\ctlf $\epsilon$\else
\if \nextchar-$-$\else
\if \nextchar\space\penalty 10000\ \else
\if \nextchar,\penalty 10000\ \else
\nextchar\spacefactor=1000
\fi\fi\fi\fi\fi\fi
\endgroup}

\def\tie{\penalty 10000\ }     % Save plain tex definition of ~.

% Define the random MIT characters

\catcode `\^^@=\active \def^^@{$\cdot$}
\catcode `\=\active \def{$\downarrow$}
\catcode `\=\active \def{$\alpha$}
\catcode `\^^C=\active \def^^C{$\beta$}
\catcode `\^^D=\active \def^^D{$\wedge$}
\catcode `\^^E=\active \def^^E{$\neg$}
\catcode `\^^G=\active \def^^G{$\pi$}
\catcode `\^^H=\active \def^^H{$\lambda$}
\catcode `\=\active \def{$\uparrow$}
\catcode `\^^L=\active \def^^L{$\plusminus$}
\catcode `\=\active \def{$\infty$}
\catcode `\=\active \def{$\partial$}
\catcode `\=\active \def{$\subset$}
\catcode `\=\active \def{$\cap$}
\catcode `\=\active \def{$\cup$}
\catcode `\=\active \def{$\forall$}
\catcode `\=\active \def{$\exists$}
\catcode `\=\active \def{$\rightleftharpoons$}
\catcode `\=\active \def{$\leftarrow$}
\catcode `\=\active \def{$\rightarrow$}
\catcode `\=\active \def{$\neq$}
\catcode `\^^[=\active \def^^[{$\diamondsuit$}
\catcode `\^^\=\active \def^^\{$\leq$}
\catcode `\^^]=\active \def^^]{$\geq$}
\catcode `\^^^=\active \def^^^{$\equiv$}
\catcode `\^^_=\active \def^^_{$\vee$}

% Font for variables

\let\vf=\sl

% Font(s) for functions being defined
\let\defsf=\twelvesf
\def\df{\let\tt=\deftt \dffont}
\def\DEFbf{\global\let\dffont=\defbf}
\def\DEFsf{\global\let\dffont=\defsf}
\DEFbf

\message{page headings,}

%%% Set up page headings and footings.

\let\thispage=\folio

\newtoks \evenheadline    % Token sequence for heading line of even pages
\newtoks \oddheadline     % Token sequence for heading line of odd pages
\newtoks \evenfootline    % Token sequence for footing line of even pages
\newtoks \oddfootline     % Token sequence for footing line of odd pages

% Now make Tex use those variables
\headline={{\textfonts\rm \ifodd\pageno \the\oddheadline \else \the\evenheadline \fi}}
\footline={{\textfonts\rm \ifodd\pageno \the\oddfootline \else \the\evenfootline \fi}}

% Commands to set those variables.
% For example, this is what  @headings on  does
% @evenheading @thistitle|@thispage|@thischapter
% @oddheading @thischapter|@thispage|@thistitle
% @evenfooting @thisfile||
% @oddfooting ||@thisfile

\def\evenheading{\parsearg\evenheadingxxx}
\def\oddheading{\parsearg\oddheadingxxx}
\def\everyheading{\parsearg\everyheadingxxx}

\def\evenfooting{\parsearg\evenfootingxxx}
\def\oddfooting{\parsearg\oddfootingxxx}
\def\everyfooting{\parsearg\everyfootingxxx}

\def\evenheadingxxx #1{\evenheadingyyy #1\|\|\|\|\finish}
\def\evenheadingyyy #1\|#2\|#3\|#4\finish{%
\global\evenheadline={\rlap{\centerline{#2}}\line{#1\hfil#3}}}

\def\oddheadingxxx #1{\oddheadingyyy #1\|\|\|\|\finish}
\def\oddheadingyyy #1\|#2\|#3\|#4\finish{%
\global\oddheadline={\rlap{\centerline{#2}}\line{#1\hfil#3}}}

\def\everyheadingxxx #1{\everyheadingyyy #1\|\|\|\|\finish}
\def\everyheadingyyy #1\|#2\|#3\|#4\finish{%
\global\evenheadline={\rlap{\centerline{#2}}\line{#1\hfil#3}}
\global\oddheadline={\rlap{\centerline{#2}}\line{#1\hfil#3}}}

\def\evenfootingxxx #1{\evenfootingyyy #1\|\|\|\|\finish}
\def\evenfootingyyy #1\|#2\|#3\|#4\finish{%
\global\evenfootline={\rlap{\centerline{#2}}\line{#1\hfil#3}}}

\def\oddfootingxxx #1{\oddfootingyyy #1\|\|\|\|\finish}
\def\oddfootingyyy #1\|#2\|#3\|#4\finish{%
\global\oddfootline={\rlap{\centerline{#2}}\line{#1\hfil#3}}}

\def\everyfootingxxx #1{\everyfootingyyy #1\|\|\|\|\finish}
\def\everyfootingyyy #1\|#2\|#3\|#4\finish{%
\global\evenfootline={\rlap{\centerline{#2}}\line{#1\hfil#3}}
\global\oddfootline={\rlap{\centerline{#2}}\line{#1\hfil#3}}}

% @headings on   turns them on.
% @headings off  turns them off.
% By default, they are off.

\def\headings #1 {\csname HEADINGS#1\endcsname}

\def\HEADINGSoff{
\global\evenheadline={\hfil} \global\evenfootline={\hfil}
\global\oddheadline={\hfil} \global\oddfootline={\hfil}}
\HEADINGSoff
% When we turn headings on, set the page number to 1,
% Put current file name in lower left corner,
% Put chapter name in upper right corner, document title in upper left,
% and page number in the center.
\def\HEADINGSon{
\global\pageno=1
\global\evenfootline={\thisfile\hfil}
\global\oddfootline={\thisfile\hfil}
\global\evenheadline={\rlap{\centerline{\folio}}\line{\thistitle\hfil\thischapter}}
\global\oddheadline={\rlap{\centerline{\folio}}\line{\thistitle\hfil\thischapter}}
}

% Subroutines used in generating headings
\def\today{\ifcase\month\or
January\or February\or March\or April\or May\or June\or
July\or August\or September\or October\or November\or December\fi
\space\number\day, \number\year}

% @settitle line...  specifies the title of the document, for headings
% It generates no output of its own

\def\thistitle{No Title}
\def\settitle{\parsearg\settitlezz}
\def\settitlezz #1{\gdef\thistitle{#1}}

\message{tables,}

% Tables -- @table, @ftable, @item(x), @kitem(x), @xitem(x).

% default indentation of table text
\newdimen\tableindent \tableindent=.8in
% default indentation of @itemize and @enumerate text
\newdimen\itemindent  \itemindent=.8in
% margin between end of table item and start of table text.
\newdimen\itemmargin  \itemmargin=.1in

% used internally for \itemindent minus \itemmargin
\newdimen\itemmax

% Note @table and @ftable define @item, @itemx, etc., with these defs.
% They also define \itemindex
% to index the item name in whatever manner is desired (perhaps none).

\def\internalBitem{\smallbreak \parsearg\itemzzz}
\def\internalBitemx{\par \parsearg\itemzzz}

\def\internalBxitem "#1"{\def\xitemsubtopix{#1} \smallbreak \parsearg\xitemzzz}
\def\internalBxitemx "#1"{\def\xitemsubtopix{#1} \par \parsearg\xitemzzz}

\def\internalBkitem{\smallbreak \parsearg\kitemzzz}
\def\internalBkitemx{\par \parsearg\kitemzzz}

\def\kitemzzz #1{\dosubind {kw}{#1}{for {\bf \lastfunction}}\itemzzz {#1}}

\def\xitemzzz #1{\dosubind {kw}{#1}{for {\bf \xitemsubtopic}}\itemzzz {#1}}

\def\itemzzz #1{\begingroup %
\advance \hsize by -\rightskip %
\advance \hsize by -\leftskip %
\setbox0=\hbox{{\itemfont #1}}%
\itemindex{#1}%
\parskip=0in %
\noindent\vadjust{\penalty 800}%
\ifdim \wd0>\itemmax %
\hbox to \hsize{\hskip -\tableindent\box0\hss}\ %
\else %
\hbox to 0pt{\hskip -\tableindent\box0\hss}%
\fi %
\endgroup %
}

\def\item{\errmessage{@item while not in a table}}
\def\itemx{\errmessage{@itemx while not in a table}}
\def\kitem{\errmessage{@kitem while not in a table}}
\def\kitemx{\errmessage{@kitemx while not in a table}}
\def\xitem{\errmessage{@xitem while not in a table}}
\def\xitemx{\errmessage{@xitemx while not in a table}}

%% Contains a kludge to get @end[description] to work
\def\description{\tablez{\dontindex}{1}{}{}{}{}}

\def\table{\begingroup\inENV\obeylines\obeyspaces\tablex}
{\obeylines\obeyspaces%
\gdef\tablex #1^^M{%
\tabley\dontindex#1        \endtabley}}

\def\ftable{\begingroup\inENV\obeylines\obeyspaces\ftablex}
{\obeylines\obeyspaces%
\gdef\ftablex #1^^M{%
\tabley\fnitemindex#1        \endtabley}}

\def\dontindex #1{}
\def\fnitemindex #1{\doind {fn}{#1}}%

{\obeyspaces %
\gdef\tabley#1#2 #3 #4 #5 #6 #7\endtabley{\endgroup%
\tablez{#1}{#2}{#3}{#4}{#5}{#6}}}

\def\tablez #1#2#3#4#5#6{%
\aboveenvbreak %
\begingroup %
\def\Edescription{\Etable}% Neccessary kludge.
\let\itemindex=#1%
\ifnum 0#3>0 \advance \leftskip by #3\mil \fi %
\ifnum 0#4>0 \tableindent=#4\mil \fi %
\ifnum 0#5>0 \advance \rightskip by #5\mil \fi %
\let\itemfont=\bf %
\edef\itemfont{\fontselect{#2}}
\itemmax=\tableindent %
\advance \itemmax by -\itemmargin %
\advance \leftskip by \tableindent %
\parindent = 0pt
\parskip = \smallskipamount
\ifdim \parskip=0pt \parskip=2pt \fi%
\def\Etable{\par\endgroup\afterenvbreak}%
\let\item = \internalBitem %
\let\itemx = \internalBitemx %
\let\kitem = \internalBkitem %
\let\kitemx = \internalBkitemx %
\let\xitem = \internalBxitem %
\let\xitemx = \internalBxitemx %
}

% This is the counter used by @enumerate, which is really @itemize

\newcount \itemno

\def\itemize{\parsearg\itemizezzz}

\def\itemizezzz #1{\itemizey {#1}{\Eitemize}}

\def\itemizey #1#2{%
\aboveenvbreak %
\begingroup %
\itemno = 0 %
\itemmax=\itemindent %
\advance \itemmax by -\itemmargin %
\advance \leftskip by \itemindent %
\parindent = 0pt
\parskip = \smallskipamount
\ifdim \parskip=0pt \parskip=2pt \fi%
\def#2{\par\endgroup\afterenvbreak}%
\def\itemcontents{#1}%
\let\item=\itemizeitem
}

\let\ptexbullet=\bullet
\def\bullet{$\ptexbullet$}

\def\enumerate{\itemizey{\the\itemno.}\Eenumerate}

% Definition of @item while inside @itemize.

\def\itemizeitem{%
\advance\itemno by 1
\par
\smallbreak 
\ifhmode \errmessage{\in hmode at itemizeitem}\fi
{\parskip=0in \hskip 0pt
\hbox to 0pt{\hss \itemcontents\hskip \itemmargin}%
\vadjust{\penalty 300}%
 }}

\message{figures,} % Floating insertions, basically
\newcount\figno \figno=0
\def\basicaption{\parsearg\captionx}
\def\captionx #1{\global\advance\figno by 1%
\par{\textfonts\line{\hfil{\bf Figure \the\figno .} #1\hfil}}}
\def\filcaption{\vfil\basicaption}
\def\iEfigure{\smallskip\hrule\endinsert}
\def\fullpagefigure{%
\pageinsert\inENV\let\caption=\filcaption\let\Efullpagefigure=\iEfigure\hrule\smallskip}
\def\figure{% This tries to be near where the command was
\midinsert\inENV\let\caption=\basicaption\let\Efigure=\iEfigure\hrule\smallskip}
\def\topfigure{% This tries to be near the top of a page
\topinsert\inENV\let\caption=\basicaption\let\Etopfigure=\iEfigure\hrule\smallskip}

\message{footnotes,}% Footnotes

\newcount \footnoteno

\def\supereject{\par\penalty -20000\footnoteno =0 }

\let\ptexfootnote=\footnote

{\catcode `\@=11
\gdef\footnote{\global\advance \footnoteno by \@ne
\edef\thisfootno{$^{\the\footnoteno}$}%
\let\@sf\empty
\ifhmode\edef\@sf{\spacefactor\the\spacefactor}\/\fi
\thisfootno\@sf\parsearg\footnotezzz}

\gdef\footnotezzz #1{\insert\footins{
\interlinepenalty\interfootnotelinepenalty
\splittopskip\ht\strutbox % top baseline for broken footnotes
\splitmaxdepth\dp\strutbox \floatingpenalty\@MM
\leftskip\z@skip \rightskip\z@skip \spaceskip\z@skip \xspaceskip\z@skip
\footstrut\hang\textindent{\thisfootno}#1\strut}}

} %end \catcode `\@=11

\message{indexing,}
% Index generation facilities

% Define \newwrite to be identical to plain tex's \newwrite
% except not \outer, so it can be used within \newindex.
{\catcode`\@=11
\gdef\newwrite{\alloc@7\write\chardef\sixt@@n}}

% \newindex {foo} defines an index named foo.
% It automatically defines \fooindex such that
% \fooindex ...rest of line... puts an entry in the index foo.
% It also defines \fooindfile to be the number of the output channel for
% the file that	accumulates this index.  The file's extension is foo.
% The name of an index should be no more than 2 characters long
% for the sake of vms.

% @defindex foo  is the form you write in a bolio file.

\def\defindex {\parsearg\newindex}

\def\newindex #1{
\expandafter\newwrite \csname#1indfile\endcsname% Define number for output file
\openout \csname#1indfile\endcsname \jobname.#1	% Open the file
\expandafter\xdef\csname#1index\endcsname{%	% Define \xxxindex
\noexpand\doindex {#1}}
}

% @synindex foo bar    makes index foo feed into index bar.
% Do this instead of @defindex foo if you don't want it as a separate index.
\def\synindex #1 #2 {%
\expandafter \xdef \csname#1index\endcsname {\noexpand\csname#2index\endcsname}%
\expandafter \chardef \csname#1indfile\endcsname %
= \csname#2indfile\endcsname}

% Define \doindex, the driver for all \fooindex macros.
% Argument #1 is generated by the calling \fooindex macro,
%  and it is "foo", the name of the index.

% \doindex just uses \parsearg; it calls \doind for the actual work.
% This is because \doind is more useful to call from other macros.

% There is also \dosubind {index}{topic}{subtopic}
% which makes an entry in a two-level index such as the operation index.

\def\doindex#1{\edef\indexname{#1}\parsearg\singleindexer}
\def\singleindexer #1{\doind{\indexname}{#1}}

\def\indexdummies{%
\def\bf{\realbackslash bf }
\def\rm{\realbackslash rm }
\def\sl{\realbackslash sl texindex.}
}

% To define \realbackslash, we must make \ not be an escape.
% We must first make another character (@) an escape
% so we do not become unable to do a definition.

{\catcode`\@=0 \catcode`\\=\other
@gdef@realbackslash{\}}

\def\doind #1#2{%
{\indexdummies  % Must do this here, since \bf, etc expand at this stage
\let\folio=0 \edef\temp{ % Expand all macros now EXCEPT \folio
\write \csname#1indfile\endcsname{%
\realbackslash entry {#2}{\folio}{#2}}}%
\temp}}

\def\dosubind #1#2#3{%
{\indexdummies  % Must do this here, since \bf, etc expand at this stage
\let\folio=0 \edef\temp{ %
\write \csname#1indfile\endcsname{%
\realbackslash entry {#2 #3}{\folio}{#2}{#3}}}%
\temp}}

% The index entry written in the file actually looks like
%  \entry {sortstring}{page}{topic}
% or
%  \entry {sortstring}{page}{topic}{subtopic}
% The texindex program reads in these files and writes files
% containing these kinds of lines:
%  \initial {c}
%     before the first topic whose initial is c
%  \entry {topic}{pagelist}
%     for a topic that is used without subtopics
%  \primary {topic}
%     for the beginning of a topic that is used with subtopics
%  \secondary {subtopic}{pagelist}
%     for each subtopic.

% Define the user-accessible indexing commands 
% @findex, @vindex, @kindex, @cindex.

\def\findex {\fnindex}
\def\kindex {\kwindex}
\def\cindex {\cpindex}
\def\vindex {\vrindex}

\def\cindexsub {\begingroup\obeylines\cindexsub}
{\obeylines %
\gdef\cindexsub "#1" #2^^M{\endgroup %
\dosubind{cp}{#2}{#1}}}

% Define the macros used in formatting output of the sorted index material.

% This is what you call to cause a particular index to get printed.
% Write
% @unnumbered Function Index
% @printindex fn

\def\printindex{\parsearg\doprintindex}

\def\doprintindex#1{\tex %
\catcode`\%=\other\catcode`\&=\other\catcode`\#=\other
\catcode`\@=\other\catcode`\$=\other\catcode`\_=\other
\catcode`\~=\other
\indexfonts\rm \tolerance=9500 \advance\baselineskip -1pt
\begindoublecolumns
\openin 1 \jobname.#1s
\ifeof 1 \else \closein 1 \input \jobname.#1s
\fi
\enddoublecolumns
\Etex}

% These macros are used by the sorted index file itself.
% Change them to control the appearance of the index.

\outer\def\initial #1{\bigbreak\line{\secbf#1\hfill}\kern 2pt\penalty3000}

\outer\def\entry #1#2{
{\parfillskip=0in \parskip=0in \parindent=0in
\hangindent=1in \hangafter=1%
\noindent\hbox{#1}\leaders\Dotsbox\hskip 0pt plus 1filll #2\par
}}

\def\primary #1{\line{#1\hfil}}

\newskip\secondaryindent \secondaryindent=0.5cm

\def\secondary #1#2{
{\parfillskip=0in \parskip=0in
\hangindent =1in \hangafter=1
\noindent\hskip\secondaryindent\hbox{#1}\leaders\Dotsbox\hskip 0pt plus 1filll#2\par
}}

%% Define two-column mode, which is used in indexes.
%% Adapted from the TeXBook, page 416
\catcode `\@=11

\newbox\partialpage

\def\begindoublecolumns{\begingroup
  \output={\global\setbox\partialpage=\vbox{\unvbox255\kern -\topskip \kern \baselineskip}}\eject
  \output={\doublecolumnout} \hsize=3.11in \vsize=19.1in}
\def\enddoublecolumns{\output={\balancecolumns}\eject
  \endgroup \pagegoal=\vsize}

\def\doublecolumnout{\splittopskip=\topskip \splitmaxdepth=\maxdepth
  \dimen@=\pageheight \advance\dimen@ by-\ht\partialpage
  \setbox0=\vsplit255 to\dimen@ \setbox2=\vsplit255 to\dimen@
  \onepageout\pagesofar \unvbox255 \penalty\outputpenalty}
\def\pagesofar{\unvbox\partialpage %
  \wd0=\hsize \wd2=\hsize \hbox to\pagewidth{\box0\hfil\box2}}
\def\balancecolumns{\setbox0=\vbox{\unvbox255} \dimen@=\ht0
  \advance\dimen@ by\topskip \advance\dimen@ by-\baselineskip
  \divide\dimen@ by2 \splittopskip=\topskip
  {\vbadness=10000 \loop \global\setbox3=\copy0
    \global\setbox1=\vsplit3 to\dimen@
    \ifdim\ht3>\dimen@ \global\advance\dimen@ by1pt \repeat}
  \setbox0=\vbox to\dimen@{\unvbox1}  \setbox2=\vbox to\dimen@{\unvbox3}
  \pagesofar}

\catcode `\@=\other
\message{sectioning,}
% Define chapters, sections, etc.

\outer\def\chapter{\parsearg\chapx}
\outer\def\unnumbered{\parsearg\chapxn}
\outer\def\appendix{\parsearg\chapapx}
\outer\def\section{\parsearg\secx}
\outer\def\appendixsection{\parsearg\apsecx}
\outer\def\unnumberedsec{\parsearg\secxn}
\outer\def\subsection{\parsearg\subsecx}
\outer\def\subsubsection{\parsearg\subsubsecx}

\newcount \chapno
\newcount \secno
\newcount \subsecno
\newcount \subsubsecno

% This counter is funny since it counts through charcodes of letters A, B, ...
\newcount \appendixno  \appendixno = `\@
\def\appendixletter{\char\the\appendixno}

\newwrite \contentsfile
\openout \contentsfile = \jobname.toc

% Each @chapter defines this as the name of the chapter.
% page headings and footings can use it.  @section does likewise

\def\thischapter{} \def\thissection{}
\def\seccheck#1{\levelcheck \if \pageno<0 %
\errmessage{@#1 not allowed after generating table of contents}\fi
%
}

\def\chapx #1{\seccheck{chapter}%
\secno=0 \subsecno=0 \subsubsecno=0 \global\advance \chapno by 1 \message{Chapter \the\chapno}%
\chapmacro {#1}{\the\chapno}%
\gdef\thissection{#1}\gdef\thischapter{#1}%
\edef\temp{{\realbackslash chapentry {#1}{\the\chapno}{\noexpand\folio}}}%
\write \contentsfile \temp  %
}

\def\chapapx #1{\seccheck{appendix}%
\secno=0 \subsecno=0 \subsubsecno=0 \global\advance \appendixno by 1 \message{Appendix \appendixletter}%
\chapmacro {#1}{Appendix \appendixletter}%
\gdef\thischapter{#1}\gdef\thissection{#1}%
\edef\temp{{\realbackslash chapentry {#1}{Appendix \appendixletter}{\noexpand\folio}}}%
\write \contentsfile \temp  %
}

\def\chapxn #1{\seccheck{unnumbered}%
\secno=0 \subsecno=0 \subsubsecno=0 \message{(#1)}
\unnumbchapmacro {#1}%
\gdef\thischapter{#1}\gdef\thissection{#1}%
\edef\temp{{\realbackslash unnumbchapentry {#1}{\noexpand\folio}}}%
\write \contentsfile \temp  %
}

\def\secx #1{\seccheck{section}%
\subsecno=0 \subsubsecno=0 \global\advance \secno by 1 %
\gdef\thissection{#1}\secheading {#1}{\the\chapno}{\the\secno}%
\edef\temp{{\realbackslash secentry %
{#1}{\the\chapno}{\the\secno}{\noexpand\folio}}}%
\write \contentsfile \temp %
}

\def\apsecx #1{\seccheck{appendixsection}%
\subsecno=0 \subsubsecno=0 \global\advance \secno by 1 %
\gdef\thissection{#1}\secheading {#1}{\appendixletter}{\the\secno}%
\edef\temp{{\realbackslash secentry %
{#1}{\appendixletter}{\the\secno}{\noexpand\folio}}}%
\write \contentsfile \temp %
}

\def\secxn #1{\seccheck{unnumberedsec}%
\plainsecheading {#1}\gdef\thissection{#1}%
\edef\temp{{\realbackslash unnumbsecentry %
{#1}{\noexpand\folio}}}%
\write \contentsfile \temp %
}

\def\subsecx #1{\seccheck{subsection}%
\gdef\thissection{#1}\subsubsecno=0 \global\advance \subsecno by 1 %
\subsecheading {#1}{\the\chapno}{\the\secno}{\the\subsecno}%
\edef\temp{{\realbackslash subsecentry %
{#1}{\the\chapno}{\the\secno}{\the\subsecno}{\noexpand\folio}}}%
\write \contentsfile \temp %
}

\def\subsubsecx #1{\seccheck{subsubsection}%
\gdef\thissection{#1}\global\advance \subsubsecno by 1 %
\subsubsecheading {#1}{\the\chapno}{\the\secno}{\the\subsecno}{\the\subsubsecno}%
\edef\temp{{\realbackslash subsubsecentry %
{#1}{\the\chapno}{\the\secno}{\the\subsecno}{\the\subsubsecno}{\noexpand\folio}}}%\
\write \contentsfile \temp %
}

% Define @majorheading, @heading and @subheading

\outer\def\majorheading{\parsearg\majorheadingzzz}

\def\majorheadingzzz #1{%
{\advance\chapheadingskip by 10pt \chapbreak }%
{\chapfonts \line{\chaprm #1\hfill}}\bigskip \par\penalty 200}

\outer\def\heading{\parsearg\headingzzz}

\def\headingzzz #1{\chapbreak %
{\chapfonts \line{\chaprm #1\hfill}}\bigskip \par\penalty 200}

\outer\def\subheading{\parsearg\secheadingi}

\outer\def\subsubheading{\parsearg\subsecheadingi}

% These macros generate a chapter, section, etc. heading only
% (including whitespace, linebreaking, etc. around it),
% given all the information in convenient, parsed form.

%%% Args are the skip and penalty (usually negative)
\def\dobreak#1#2{\par\ifdim\lastskip<#1\removelastskip\penalty#2\vskip#1 \fi}

\def\setchapterstyle #1 {\csname CHAPF#1\endcsname}

%%% Define plain chapter starts, and page on/off switching for it
% Parameter controlling skip before chapter headings (if needed)

\newskip \chapheadingskip \chapheadingskip = 30pt plus 8pt minus 4pt

\def\chapbreak{\dobreak \chapheadingskip {-100}}
\def\chappager{\par\vfill\supereject}
\def\chapoddpage{\chappager \ifodd\pageno \else \hbox to 0pt{} \chappager\fi}

\def\setchapternewpage #1 {\csname CHAPPAG#1\endcsname}
\def\CHAPPAGoff{\global\let\pchapsepmacro=\chapbreak}
\def\CHAPPAGon{\global\let\pchapsepmacro=\chappager}
\def\CHAPPAGodd{\global\let\pchapsepmacro=\chapoddpage}
\CHAPPAGon

\def\CHAPFplain{
\global\let\chapmacro=\chfplain
\global\let\unnumbchapmacro=\unnchfplain}

\def\chfplain #1#2{%
\pchapsepmacro %
{\chapfonts \line{\chaprm #2.\enspace #1\hfill}}\bigskip \par\penalty 5000 %
}

\def\unnchfplain #1{%
\pchapsepmacro %
{\chapfonts \line{\chaprm #1\hfill}}\bigskip \par\penalty 10000 %
}
\CHAPFplain % The default

\def\unnchfopen #1{%
\chapoddpage {\chapfonts \line{\chaprm #1\hfill}}\bigskip \par\penalty 10000 %
}

\def\chfopen #1#2{\chapoddpage {\chapfonts
\vbox to 3in{\vfil \hbox to\hsize{\hfil #2} \hbox to\hsize{\hfil #1} \vfil}}%
\par\penalty 5000 %
}

\def\CHAPFopen{
\global\let\chapmacro=\chfopen
\global\let\unnumbchapmacro=\unnchfopen}

% Parameter controlling skip before section headings.

\newskip \subsecheadingskip  \subsecheadingskip = 17pt plus 4pt minus 4pt
\def\subsecheadingbreak{\dobreak \subsecheadingskip {-30}}

\newskip \secheadingskip  \secheadingskip = 21pt plus 6pt minus 4pt
\def\secheadingbreak{\dobreak \secheadingskip {-50}}

\def\secheading #1#2#3{\secheadingi {#2.#3\enspace #1}}
\def\plainsecheading #1{\secheadingi {#1}}
\def\secheadingi #1{{\advance \secheadingskip by \parskip %
\secheadingbreak} %
{\secfonts \line{\secrm #1\hfill}} %
\ifdim \parskip<10pt \kern 10pt\kern -\parskip\fi \penalty 1000}

\def\subsecheading #1#2#3#4{{\advance \subsecheadingskip by \parskip %
\subsecheadingbreak} %
{\subsecfonts \line{\secrm#2.#3.#4\enspace #1\hfill}}
\ifdim \parskip<10pt \kern 10pt\kern -\parskip\fi \penalty 1000}

\def\subsubsecfonts{\subsecfonts} % Maybe this should change

\def\subsubsecheading #1#2#3#4#5{{\advance \subsecheadingskip by \parskip %
\subsecheadingbreak} %
{\subsubsecfonts \line{\secrm#2.#3.#4.#5\enspace #1\hfill}}
\ifdim \parskip<10pt \kern 10pt\kern -\parskip\fi \penalty 1000}

\message{toc printing,}

\def\Dotsbox{\hbox to 1em{\hss.\hss}} % Used by index macros

\def\finishcontents{%
\ifnum\pageno>0 %
\par\vfill\supereject %
\immediate\closeout \contentsfile%
\pageno=-1		% Request roman numbered pages
\fi}

\outer\def\contents{%
\finishcontents %
\unnumbchapmacro{Table of Contents}
\def\thischapter{Table of Contents}
{\catcode`\\=0
\catcode`\{=1		% Set up to handle contents files properly
\catcode`\}=2
\input \jobname.toc
}
\vfill \eject}

\outer\def\summarycontents{%
\finishcontents %
\unnumbchapmacro{Summary Table of Contents}
\def\thischapter{Summary Table of Contents}
{\catcode`\\=0
\catcode`\{=1		% Set up to handle contents files properly
\catcode`\}=2
\def\smallbreak{}
\def\secentry ##1##2##3##4{}
\def\subsecentry ##1##2##3##4##5{}
\def\subsubsecentry ##1##2##3##4##5##6{}
\def\unnumbsecentry ##1##2{}
\let\medbreak=\smallbreak
\input \jobname.toc
}
\vfill \eject}

\outer\def\bye{\par\vfill\supereject\tracingstats=1\ptexend}

% These macros generate individual entries in the table of contents
% The first argument is the chapter or section name.
% The last argument is the page number.
% The arguments in between are the chapter number, section number, ...

\def\chapentry #1#2#3{%
\medbreak
\line{#2.\space#1\leaders\hbox to 1em{\hss.\hss}\hfill #3}
}

\def\unnumbchapentry #1#2{%
\medbreak
\line{#1\leaders\Dotsbox\hfill #2}
}

\def\secentry #1#2#3#4{%
\line{\enspace\enspace#2.#3\space#1\leaders\Dotsbox\hfill#4}
}

\def\unnumbsecentry #1#2{%
\line{\enspace\enspace#1\leaders\Dotsbox\hfill #2}
}

\def\subsecentry #1#2#3#4#5{%
\line{\enspace\enspace\enspace\enspace
#2.#3.#4\space#1\leaders\Dotsbox\hfill #5}
}

\def\subsubsecentry #1#2#3#4#5#6{%
\line{\enspace\enspace\enspace\enspace\enspace\enspace
#2.#3.#4.#5\space#1\leaders\Dotsbox\hfill #6}
}

\message{environments,}

% @tex ... @end tex    escapes into raw Tex temporarily.

\def\tex{\begingroup
\catcode `\\=0 \catcode `\{=1 \catcode `\}=2
\catcode `\$=3 \catcode `\&=4 \catcode `\#=6
\catcode `\^=7 \catcode `\_=8 \catcode `\~=13 \let~=\tie
\let\{=\lbrace \let\}=\rbrace
\let\nobreak=\ptexnobreak
\let\.=\ptexdot
\let\#=\ptexnumsign
\let\*=\ptexstar
\let\+=\tabalign
\let\-=\ptexminus
\let\b=\ptexb \let\c=\ptexc \let\i=\ptexi \let\t=\ptext \let\l=\ptexl
\let\L=\ptexL
\catcode `\%=14 \let\Etex=\endgroup}

% Define @lisp ... @endlisp.
% @lisp does a \begingroup so it can rebind things,
% including the definition of @endlisp (which normally is erroneous).

% Amount to narrow the margins by for @lisp.
\newskip\lispnarrowing \lispnarrowing=0.3in

% This is the definition that ^M gets inside @lisp
{\obeyspaces%
\gdef\lisppar{ \endgraf}}

% Cause \obeyspaces to make each Space cause a word-separation
% rather than the default which is that it acts punctuation.
% This is because space in tt font looks funny.
{\obeyspaces %
\gdef\sepspaces{\def {\ }}}

\newskip\aboveenvskipamount \aboveenvskipamount=3pt
\def\aboveenvbreak{{\advance\aboveenvskipamount by \parskip
\par \ifdim\lastskip<\aboveenvskipamount
\removelastskip \penalty-50 \vskip\aboveenvskipamount \fi}}

\def\afterenvbreak{\par \ifdim\lastskip<\aboveenvskipamount
\removelastskip \penalty-50 \vskip\aboveenvskipamount \fi}

{\catcode`\@=\active%
% @ must be active when this definition is made,
% so that the \let@ can be parsed properly.
\gdef\lisp{\begingroup\inENV %This group ends at the end of the @lisp body
\aboveenvbreak \hfuzz=12truept % Don't be fussy
% Make spaces be word-separators rather than space tokens.
\sepspaces %
% Single space lines
\singlespace %
% The following causes blank lines not to be ignored
% by adding a space to the end of each line.
\let\par=\lisppar
\catcode`\{=\other \catcode`\}=\other
\def\Elisp{\endgroup\afterenvbreak}
\parskip=0pt \advance \rightskip by \lispnarrowing 
\advance \leftskip by \lispnarrowing
\parindent=0pt
\let\exdent=\internalexdent
\obeyspaces \obeylines \tt}}

\let\example=\lisp
\def\Eexample{\Elisp}

% This is @display; same as @lisp except use roman font.

{\catcode`\@=\active%
% @ must be active when this definition is made,
% so that the \let@ can be parsed properly.
\gdef\display{\begingroup\inENV %This group ends at the end of the @display body
\aboveenvbreak
% Make spaces be word-separators rather than space tokens.
\sepspaces %
% Single space lines
\singlespace %
% The following causes blank lines not to be ignored
% by adding a space to the end of each line.
\let\par=\lisppar
\def\Edisplay{\endgroup\afterenvbreak}
\parskip=0pt \advance \rightskip by \lispnarrowing 
\advance \leftskip by \lispnarrowing
\parindent=0pt
\let\exdent=\internalexdent
\obeyspaces \obeylines}}

% This is @format; same as @lisp except use roman font and don't narrow margins

{\catcode`\@=\active%
% @ must be active when this definition is made,
% so that the \let@ can be parsed properly.
\gdef\format{\begingroup\inENV %This group ends at the end of the @format body
\aboveenvbreak
% Make spaces be word-separators rather than space tokens.
\sepspaces %
\singlespace %
% The following causes blank lines not to be ignored
% by adding a space to the end of each line.
\let\par=\lisppar
\def\Eformat{\endgroup\afterenvbreak}
\parskip=0pt \parindent=0pt
\obeyspaces \obeylines}}

% This is @address; same as @format except put left margin in mid page

{\catcode`\@=\active%
% @ must be active when this definition is made,
% so that the \let@ can be parsed properly.
\gdef\address{\begingroup\inENV %This group ends at the end of the @address body
\aboveenvbreak
% Make spaces be word-separators rather than space tokens.
\sepspaces %
% The following causes blank lines not to be ignored
% by adding a space to the end of each line.
% This also causes @ to work when the directive name
% is terminated by end of line.
\let\par=\lisppar
\def\Eaddress{\endgroup\afterenvbreak}
\def\Eclosing{\endgroup\afterenvbreak}
\parskip=0pt \parindent=0pt
\obeyspaces \obeylines}}

\let\closing=\address

% @flushleft and @flushright

{\catcode`\@=\active%
% @ must be active when this definition is made,
% so that the \let@ can be parsed properly.
\gdef\flushleft{\begingroup\inENV %This group ends at the end of the @format body
\aboveenvbreak
% Make spaces be word-separators rather than space tokens.
\sepspaces %
% The following causes blank lines not to be ignored
% by adding a space to the end of each line.
% This also causes @ to work when the directive name
% is terminated by end of line.
\let\par=\lisppar
\def\Eflushleft{\endgroup\afterenvbreak}
\parskip=0pt \parindent=0pt
\obeyspaces \obeylines}}

{\catcode`\@=\active%
% @ must be active when this definition is made,
% so that the \let@ can be parsed properly.
\gdef\flushright{\begingroup\inENV %This group ends at the end of the @format body
\aboveenvbreak
% Make spaces be word-separators rather than space tokens.
\sepspaces %
% The following causes blank lines not to be ignored
% by adding a space to the end of each line.
% This also causes @ to work when the directive name
% is terminated by end of line.
\let\par=\lisppar
\def\Eflushright{\endgroup\afterenvbreak}
\parskip=0pt \parindent=0pt
\advance \leftskip by 0pt plus 1fill
\obeyspaces \obeylines}}

% @quotation - narrow the margins.

\def\quotation{\begingroup\inENV %This group ends at the end of the @quotation body
\aboveenvbreak
\singlespace
\def\Equotation{\par\endgroup\afterenvbreak}
\advance \rightskip by \lispnarrowing 
\advance \leftskip by \lispnarrowing}

% @undent - make every paragraph have a hanging indentation

\def\undent{\begingroup %This group ends at the end of the @undent body
\def\Eundent{\par\endgroup}
\everypar={\hangindent=\parindent \hskip-\parindent \hangafter=1 }}

% @document - nothing now, later it will read in the aux file, close toc files
\def\document{} \def\Edocument{}

\message{defuns,}
% Define formatter for defuns
% First, allow user to change definition object font (\df) internally
\def\setdeffont #1 {\csname DEF#1\endcsname}

\newskip\defbodyindent \defbodyindent=36pt
\newskip\defargsindent \defargsindent=50pt
\newskip\deftypemargin \deftypemargin=12pt
\newskip\deflastargmargin \deflastargmargin=18pt

\newcount\parencount
% define \functionparens, which makes ( and ) and & do special things.
% \functionparens affects the group it is contained in.
\def\activeparens{%
\catcode`\(=\active \catcode`\)=\active \catcode`\&=\active
\catcode`\[=\active \catcode`\]=\active}
{\activeparens % Now, smart parens don't turn on until &foo (see \amprm)
\gdef\functionparens{\boldbrax\let&=\amprm\parencount=0 }
\gdef\boldbrax{\let(=\opnr\let)=\clnr\let[=\lbrb\let]=\rbrb}

% Definitions of (, ) and & used in args for functions.
% This is the definition of ( outside of all parentheses.
\gdef\oprm#1 {{\rm\char`\(}#1 \bf \let(=\opnested %
\global\advance\parencount by 1 }
%
% This is the definition of ( when already inside a level of parens.
\gdef\opnested{\char`\(\global\advance\parencount by 1 }
%
\gdef\clrm{% Print a paren in roman if it is taking us back to depth of 0.
% also in that case restore the outer-level definition of (.
\ifnum \parencount=1 {\rm \char `\)}\vf \let(=\oprm \else \char `\) \fi
\global\advance \parencount by -1 }
% If we encounter &foo, then turn on ()-hacking afterwards
\gdef\amprm#1 {{\rm\&#1}\let(=\oprm \let)=\clrm\ }
%
\gdef\normalparens{\boldbrax\let&=\ampnr}
} % End of definition inside \activeparens
%% These parens (in \boldbrax) actually are a little bolder than the
%% contained text.  This is especially needed for [ and ]
\def\opnr{{\sf\char`\(}} \def\clnr{{\sf\char`\)}} \def\ampnr{\&}
\def\lbrb{{\tt\char`\[}} \def\rbrb{{\tt\char`\]}}

% First, defname, which formats the header line itself.
% #1 should be the function name.
% #2 should be the type of definition, such as "Function".

\def\defname #1#2{%
\leftskip = 0in  %
\noindent        %
\setbox0=\hbox{\hskip \deflastargmargin{\rm #2}\hskip \deftypemargin}%
\dimen0=\hsize \advance \dimen0 by -\wd0 % compute size for first line
\dimen1=\hsize \advance \dimen1 by -\defargsindent %size for continuations
\parshape 2 0in \dimen0 \defargsindent \dimen1     %
% Now output arg 2 ("Function" or some such)
% ending at \deftypemargin from the right margin,
% but stuck inside a box of width 0 so it does not interfere with linebreaking
\rlap{\rightline{{\rm #2}\hskip \deftypemargin}}%
\tolerance=10000 \hbadness=10000    % Make all lines underfull and no complaints
{\df #1}\enskip        % Generate function name
}

% Actually process the body of a definition
% #1 should be the terminating control sequence, such as \Edefun.
% #2 should be the "another name" control sequence, such as \defunx.
% #3 should be the control sequence that actually processes the header,
%    such as \defunheader.

\def\defparsebody #1#2#3{\begingroup\inENV% Environment for definitionbody
\medbreak %
% Define the end token that this defining construct specifies
% so that it will exit this group.
\def#1{\endgraf\endgroup\medbreak}%
\def#2{\begingroup\obeylines\activeparens\spacesplit#3}%
\parindent=0in \leftskip=\defbodyindent %
\begingroup\obeylines\activeparens\spacesplit#3}

\def\defmethparsebody #1#2#3#4 {\begingroup\inENV %
\medbreak %
% Define the end token that this defining construct specifies
% so that it will exit this group.
\def#1{\endgraf\endgroup\medbreak}%
\def#2##1 {\begingroup\obeylines\activeparens\spacesplit{#3{##1}}}%
\parindent=0in \leftskip=\defbodyindent %
\begingroup\obeylines\activeparens\spacesplit{#3{#4}}}

% Split up #2 at the first space token.
% call #1 with two arguments:
%  the first is all of #2 before the space token,
%  the second is all of #2 after that space token.
% If #2 contains no space token, all of it is passed as the first arg
% and the second is passed as empty.

{\obeylines
\gdef\spacesplit#1#2^^M{\endgroup\spacesplitfoo{#1}#2 \relax\spacesplitfoo}%
\long\gdef\spacesplitfoo#1#2 #3#4\spacesplitfoo{%
\ifx\relax #3%
#1{#2}{}\else #1{#2}{#3#4}\fi}}

% So much for the things common to all kinds of definitions.

% Define \defun.

% First, define the processing that is wanted for arguments of \defun
% Use this to expand the args and terminate the paragraph they make up

\def\defunargs #1{\functionparens \vf #1%
\ifnum\parencount=0 \else \errmessage{unbalanced parens in @def arguments}\fi%
\interlinepenalty=10000
\endgraf\vskip -\parskip \penalty 10000}

% \lastfunction is always defined to the name in the most recennt @defun
% It is used by @kitem as the subtopic to index the keyword under

\def\lastfunction{}

% Do complete processing of one @defun or @defunx line already parsed.

\def\defunheader #1#2{\doind {fn}{#1} % Make entry in function index
\gdef\lastfunction{#1}%
\dosetq {#1-fun}{page}%
{\defname {#1}{Function}\defunargs {#2}}%
}

\def\defmacheader #1#2{\doind {fn}{#1} % Make entry in function index
\gdef\lastfunction{#1}%
\dosetq {#1-fun}{page}%
\begingroup\defname {#1}{Macro}%
\defunargs {#2}\endgroup %
}

\def\defspecheader #1#2{\doind {fn}{#1} % Make entry in function index
\gdef\lastfunction{#1}%
\dosetq {#1-fun}{page}%
\begingroup\defname {#1}{Special form}%
\defunargs {#2}\endgroup %
}

\def\defmessageheader #1#2{\doind {op}{#1} % Make entry in operation index
\gdef\lastfunction{#1}%
\dosetq {#1-message}{page}%
\begingroup\defname {#1}{Operation}%
\defunargs {#2}\endgroup %
}

% Now we can define @defun itself.

\def\defun{\defparsebody\Edefun\defunx\defunheader}
\def\defmac{\defparsebody\Edefmac\defmacx\defmacheader}
\def\defspec{\defparsebody\Edefspec\defspecx\defspecheader}
\def\defmessage{\defparsebody\Edefmessage\defmessagex\defmessageheader}

% This definition is run if you use @defunx
% anywhere other than immediately after a @defun or @defunx.

\def\defunx #1 {\errmessage{@defunx in invalid context}}
\def\defmacx #1 {\errmessage{@defmacx in invalid context}}
\def\defspecx #1 {\errmessage{@defspecx in invalid context}}
\def\defmessagex #1 {\errmessage{@defmessagex in invalid context}}

% @defmethod, @defmetamethod, and so on

% Do complete processing of one @defmethod line already parsed.

\def\defmethodheader #1#2#3{\dosubind {op}{#2}{on {\bf #1}}% Make entry in operation index
\gdef\lastfunction{#2}%
\dosetqflushcolon {#1-}#2-method {page}%
\begingroup\defname {#2}{Operation on {\bf #1}}%
\defunargs {#3}\endgroup %
}

\def\defmetamethodheader #1#2#3{\dosubind {op}{#2}{on #1}% Make entry in operation index
\gdef\lastfunction{#2}%
\dosetqflushcolon {#1-}#2-method {page}%
\begingroup\defname {#2}{Operation on #1}%
\defunargs {#3}\endgroup %
}

\def\defivarheader #1#2#3{%
\gdef\lastfunction{#2}%
\dosubind {iv}{#2}{of {\bf #1}}% Make entry in instance variable index
\dosetq {#1-#2-ivar}{page}%
\begingroup\defname {#2}{Instance variable of {\bf #1}}%
\defvarargs {#3}\endgroup %
}

\def\defmetaivarheader #1#2#3{%
\gdef\lastfunction{#2}%
\dosubind {iv}{#2}{of #1}% Make entry in ivar index
\dosetq {#1-#2-ivar}{page}%
\begingroup\defname {#2}{Instance variable of #1}%
\defvarargs {#3}\endgroup %
}

\def\definitheader #1#2#3{%
\gdef\lastfunction{#2}%
\dosubind {io}{#2}{for {\bf #1}}% Make entry in init option index
\dosetqflushcolon {#1-}#2-init-option {page}%
\begingroup\defname {#2}{Init keyword for {\bf #1}}%
\defvarargs {#3}\endgroup %
}

\def\defmetainitheader #1#2#3{%
\gdef\lastfunction{#2}%
\dosubind {io}{#2}{for #1}% Make entry in init option index
\dosetqflushcolon {#1-}#2-init-option {page}%
\begingroup\defname {#2}{Init keyword for #1}%
\defvarargs {#3}\endgroup %
}

\def\defmethod{\defmethparsebody\Edefmethod\defmethodx\defmethodheader}
\def\defmetamethod{%
\defmethparsebody\Edefmetamethod\defmetamethodx\defmetamethodheader}

\def\defivar{\defmethparsebody\Edefivar\defivarx\defivarheader}
\def\defmetaivar{%
\defmethparsebody\Edefmetaivar\defmetaivarx\defmetaivarheader}

\def\definit{\defmethparsebody\Edefinit\definitx\definitheader}
\def\defmetainit{%
\defmethparsebody\Edefmetainit\defmetainitx\defmetainitheader}

% These definitions are run if you use @defmethodx, etc.,
% anywhere other than immediately after a @defmethod, etc.

\def\defmethodx #1 {\errmessage{@defmethodx in invalid context}}
\def\defmetamethodx #1 {\errmessage{@defmetamethodx in invalid context}}

\def\defivarx #1 {\errmessage{@defivarx in invalid context}}
\def\defmetaivarx #1 {\errmessage{@defmetaivarx in invalid context}}

\def\definitx #1 {\errmessage{@definitx in invalid context}}
\def\defmetainitx #1 {\errmessage{@defmetainitx in invalid context}}

% Now @defvar

% First, define the processing that is wanted for arguments of @defvar.
% This is actually simple: just print them in roman.
% This must expand the args and terminate the paragraph they make up
\def\defvarargs #1{\normalparens #1%
\interlinepenalty=10000
\endgraf\vskip -\parskip \penalty 10000}

% Do complete processing of one @defvar or @defvarx line already parsed.

\def\defvarheader #1#2{\doind {vr}{#1}% Make entry in var index
\gdef\lastfunction{#1}%
\dosetq {#1-var}{page}%
\begingroup\defname {#1}{Variable}%
\defvarargs {#2}\endgroup %
}

\def\defconstheader #1#2{\doind {vr}{#1}% Make entry in var index
\gdef\lastfunction{#1}%
\dosetq {#1-var}{page}%
\begingroup\defname {#1}{Constant}%
\defvarargs {#2}\endgroup %
}

\def\defmeterheader #1#2{\doind {mt}{#1}% Make entry in meter index
\gdef\lastfunction{#1}%
\dosetq {#1-meter}{page}%
\begingroup\defname {#1}{Meter}%
\defvarargs {#2}\endgroup %
}

\def\defresourceheader #1#2{\doind {rs}{#1}% Make entry in resource index
\gdef\lastfunction{#1}%
\dosetq {#1-resource}{page}%
\begingroup\defname {#1}{Resource}%
\defvarargs {#2}\endgroup %
}

% Now we can define @defvar itself.  Also @defconst.

\def\defvar{\defparsebody\Edefvar\defvarx\defvarheader}
\def\defconst{\defparsebody\Edefconst\defconstx\defconstheader}
\def\defmeter{\defparsebody\Edefmeter\defmeterx\defmeterheader}
\def\defresource{\defparsebody\Edefresource\defresourcex\defresourceheader}

% This definition is run if you use @defvarx
% anywhere other than immediately after a @defvar or @defvarx.

\def\defvarx #1 {\errmessage{@defvarx in invalid context}}
\def\defconstx #1 {\errmessage{@defconstx in invalid context}}
\def\defmeterx #1 {\errmessage{@defmeterx in invalid context}}
\def\defresourcex #1 {\errmessage{@defresourcex in invalid context}}

% Now define @defflavor
% Args are printed in bold, a slight difference from @defvar.

\def\defflavargs #1{\bf \defvarargs{#1}}

% Do complete processing of one @defflavor (or similar construct) line.

\def\defflavorheader #1#2{\doind {fl}{#1}% Make entry in flavor index
\gdef\lastfunction{#1}%
\dosetq {#1-flavor}{page}%
\begingroup\defname {#1}{Flavor}%
\defflavargs {#2}\endgroup %
}

\def\defconditionheader #1#2{\doind {cn}{#1}% Make entry in condition index
\gdef\lastfunction{#1}%
\dosetq {#1-condition}{page}%
\begingroup\defname {#1}{Condition}%
\defflavargs {#2}\endgroup %
}

\def\defconditionflavorheader #1#2{%
\gdef\lastfunction{#1}%
\doind {cn}{#1}% Make entry in condition index
\doind {fl}{#1}% Make entry in flavor index
\dosetq {#1-condition}{page}%
\dosetq {#1-flavor}{page}%
\dosetq {#1-condition-flavor}{page}%
\doind {fl}{#1}% Make entry in flavor index
\begingroup\defname {#1}{Condition flavor}%
\defflavargs {#2}\endgroup %
}

\def\defflavor{\defparsebody\Edefflavor\defflavorx\defflavorheader}
\def\defcondition{%
\defparsebody\Edefcondition\defconditionx\defconditionheader}
\def\defconditionflavor{%
\defparsebody\Edefconditionflavor\defconditionflavorx\defconditionflavorheader}

% This definition is run if you use @defflavorx, etc
% anywhere other than immediately after a @defflavor, etc.

\def\defflavorx #1 {\errmessage{@defflavorx in invalid context}}
\def\defconditionx #1 {\errmessage{@defconditionx in invalid context}}
\def\defconditionflavorx #1 {\errmessage{@defconditionflavorx in invalid context}}

\message{cross reference,}
% Define cross-reference macros
\newwrite \auxfile

% Define @setq.  @setq foo page  defines a string-variable named foo
%  whose value is "page nnn".
% Also allowed are
% @setq foo section-page      defines as "section c.s.ss, page nnn"
% @setq foo page-number       defines as "nnn"
% @setq foo chapter-number    defines as "c"
% @setq foo section-number    defines as "c.s.ss"

% @setx foo text              defines foo as "text", literally.

% Turn on \obeylines before parsing the arguments.
% \setqx does the actual parsing.
\def\setq{\begingroup\obeylines \setqx}

\def\setx{\begingroup\obeylines \setxx}

% Define \setqx, which just puts the arguments into a \write
% preceded by \internalsetq, which will expand at write time and do all the real work,

{ \obeylines %
\gdef\setqx #1 #2^^M{%
\let\folio=0\edef\next{\write\auxfile{\internalsetq {#1}{#2}}}%
\next\endgroup}
%
\gdef\setxx #1 #2^^M{%
\write\auxfile{'xrdef {#1}{#2}}\endgroup}}

%% @label for Scribe.  setq's x-pg, x-title, x-snam, x for
%% @pageref, @title, @nameref, @ref
\def\label{\parsearg\labelx} \def\pageref{\parsearg\pagerefx}
\def\nameref{\parsearg\namerefx}
\def\labelx#1{\dosetq{#1-pg}{page-number}\dosetq{#1-title}{title}%
\dosetq{#1-snam}{section-name}\dosetq{#1}{section-number}}
\def\pagerefx#1{\refx{#1-pg}} \def\namerefx#1{\refx{#1-snam}}
\def\titlex#1{\refx{#1-title}}

% \dosetq is the interface for calls from other macros

\def\dosetq #1#2{{\let\folio=0%
\edef\next{\write\auxfile{\internalsetq {#1}{#2}}}%
\next}}

% dosetqflushcolon {foo-}:bar {value} defines foo-bar to value,
% thus flushing the colon from the front of :bar.
% This is how we define the variables for @defmethod, etc.

\def\dosetqflushcolon #1:#2 #3{\write\auxfile{\internalsetq {#1#2}{#3}}}

% \internalsetq {foo}{page} expands into CHARACTERS 'xrdef {foo}{...expansion of \Ypage...}
% When the aux file is read, ' is the escape character

\def\internalsetq #1#2{'xrdef {#1}{\csname Y#2\endcsname}}

% We want to do \edef\Xfoo{\Ypage}
% Define \Ypage to generate "page nnn".

\def\Ypage{page \folio}
\def\Ytitle{\ifnum\secno=0 chapter\else section\fi}
% Define \Ysection-page to generate "section m.n, page n"
% Define \Ysection-number-and-type to generate "section m.n"
% Define \Ysection-number to generate just "m.n"

{\catcode `\-=11
\gdef\Ysection-page{%
\ifnum\secno=0 chapter\xreftie\the\chapno, page\xreftie\folio %
\else \ifnum \subsecno=0 section\xreftie\the\chapno.\the\secno, page\xreftie\folio %
\else \ifnum \subsubsecno=0 %
section\xreftie\the\chapno.\the\secno.\the\subsecno, page\xreftie\folio %
\else %
section\xreftie\the\chapno.\the\secno.\the\subsecno.\the\subsubsecno, page\xreftie\folio %
\fi \fi \fi}

\gdef\Ysection-number-and-type{%
\ifnum\secno=0 chapter\xreftie\the\chapno %
\else \ifnum \subsecno=0 section\xreftie\the\chapno.\the\secno %
\else \ifnum \subsubsecno=0 %
section\xreftie\the\chapno.\the\secno.\the\subsecno %
\else %
section\xreftie\the\chapno.\the\secno.\the\subsecno.\the\subsubsecno %
\fi \fi \fi }

\gdef\xreftie{'tie}

\gdef\Ysection-number{%
\ifnum\secno=0 \the\chapno %
\else \ifnum \subsecno=0 \the\chapno.\the\secno %
\else \ifnum \subsubsecno=0 %
\the\chapno.\the\secno.\the\subsecno %
\else %
\the\chapno.\the\secno.\the\subsecno.\the\subsubsecno %
\fi \fi \fi}}

% Define \Ychapter-number, \Ypage-number, \Ychapter-name \Ysection-name
{\catcode `\-=11
\gdef\Ychapter-number{\the\chapno}
\gdef\Ypage-number{\folio}
\gdef\Ychapter-name{\thischapter}
\gdef\Ysection-name{\thissection}
}

% Define @ref, and alternatively the character ^V, to reference a cross-ref.
\def\ref{\parsearg\refx}
\def\refx#1{%
{%
\setbox0=\hbox{\csname X#1\endcsname}%
\ifdim\wd0>0in \else  	% If not defined, say something at least.
\expandafter\gdef\csname X#1\endcsname {$<$undefined$>$}%
\message {WARNING: Cross-reference "#1" used but not yet defined}%
\message {}%
\fi %
\csname X#1\endcsname %It's defined, so just use it.
}}

%% Mention a Lisp construct, telling the reader the page number
%% The first argument is the construct, the second the name of the construct
%% Sample use: @see[cons][fun] --> cons (see page 23)
\def\see[#1][#2]{{\li #1} (see \refx{#1-#2})}

\catcode`\^^V=\active
\let^^V=\ref

% Read the last existing aux file, if any.  No error if none exists.

% This is the macro invoked by entries in the aux file.
\def\xrdef #1#2{
{\catcode`\'=\other\expandafter \gdef \csname X#1\endcsname {#2}}}

{
\catcode `\^^@=\other
\catcode `\=\other
\catcode `\=\other
\catcode `\^^C=\other
\catcode `\^^D=\other
\catcode `\^^E=\other
\catcode `\^^F=\other
\catcode `\^^G=\other
\catcode `\^^H=\other
\catcode `\=\other
\catcode `\^^L=\other
\catcode `\=\other
\catcode `\=\other
\catcode `\=\other
\catcode `\=\other
\catcode `\=\other
\catcode `\=\other
\catcode `\=\other
\catcode `\=\other
\catcode `\=\other
\catcode `\=\other
\catcode `\=\other
\catcode `\=\other
\catcode `\=\other
\catcode `\^^[=\other
\catcode `\^^\=\other
\catcode `\^^]=\other
\catcode `\^^^=\other
\catcode `\^^_=\other
\catcode `\@=\other
\catcode `\^=\other
\catcode `\~=\other
\catcode `\[=\other
\catcode `\]=\other
\catcode`\"=\other
\catcode`\_=\other
\catcode`\|=\other
\catcode`\<=\other
\catcode`\>=\other
\catcode `\$=\other
\catcode `\#=\other
\catcode `\&=\other

% the aux file uses ' as the escape.
% Turn off \ as an escape so we do not lose on
% entries which were dumped with control sequences in their names.
% For example, 'xrdef {$\leq $-fun}{page ...} made by @defun ^^
% Reference to such entries still does not work the way one would wish,
% but at least they do not bomb out when the aux file is read in.

\catcode `\{=1 \catcode `\}=2
\catcode `\%=\other
\catcode `\'=0
\catcode `\\=\other

'openin 1 'jobname.aux
'ifeof 1 'else 'closein 1 'input 'jobname.aux
'fi
}

% Open the new aux file.  Tex will close it automatically at exit.

\openout \auxfile=\jobname.aux

% End of control word definitions.

\message{and turning on Bolio.}

% Turn off all special characters except @
% (and those which the user can use as if they were ordinary)
% Define certain chars to be always in tt font.

\catcode`\"=\active
\def\activedoublequote{{\tt \char '042}}
\let"=\activedoublequote
\catcode`\~=\active
\def~{{\tt \char '176}}
\chardef\hat=`\^
\catcode`\^=\active
\def^{{\tt \hat}}
\catcode`\_=\active
\def_{{\tt \char '137}}
\catcode`\|=\active
\def|{{\tt \char '174}}
\chardef \less=`\<
\catcode`\<=\active
\def<{{\tt \less}}
\chardef \gtr=`\>
\catcode`\>=\active
\def>{{\tt \gtr}}

%%% CHANGE! Used by \activeparens for \defspec's benefit
\def\nlbrace{\ifmmode \delimiter"4266308 \else {\tensy\char'146}\fi}
\def\nrbrace{\ifmmode \delimiter"5267309 \else {\tensy\char'147}\fi}
 \let\{=\nlbrace \let \}=\nrbrace

\def\mylbrace {{\tt \char '173}}
\def\myrbrace {{\tt \char '175}}

\catcode`\{=\active
\let{=\mylbrace
\catcode`\}=\active
\let}=\myrbrace

%% These look ok in all fonts, so just make them not special.  The @rm below
%% makes sure that the current font starts out as the newly loaded cmr10
\catcode`\$=\other \catcode`\%=\other \catcode`\&=\other \catcode`\#=\other
\catcode`\@=0 \catcode`\\=\other

@textfonts
@rm
